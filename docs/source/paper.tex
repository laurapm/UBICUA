%%%%%%%%%%%%%%%%%%%%%%%%%%%%%%%%%%%%%%%%%%%%%%%%%%%%%%%%%%%%%%%%%%%%%%%%%%%%%%%%
% Ubiquitous Computation - Lab Practice 1                                      %
%   - Pablo Acereda García                                                     %
%   - David Emanuel Craciunescu                                                %
%   - Pablo Martínez Gracia                                                    %
%   - Laura Pérez Medeiro                                                      %
%%%%%%%%%%%%%%%%%%%%%%%%%%%%%%%%%%%%%%%%%%%%%%%%%%%%%%%%%%%%%%%%%%%%%%%%%%%%%%%%

% This template has been tested with LLNCS DOCUMENT CLASS -- version 2.20 (10-Mar-2018)

% !TeX spellcheck   = en-US
% !TeX encoding     = utf8
% !TeX program      = pdflatex
% !BIB program      = bibtex
% -*- coding:utf-8 mod:LaTeX -*-

% "a4paper" enables:
%  - easy print out on DIN A4 paper size
%

% English documents: pass english as class option
\documentclass[english,runningheads,a4paper]{llncs}[2018/03/10]
\usepackage[ngerman,main=english]{babel}
\addto\extrasenglish{\languageshorthands{ngerman}\useshorthands{"}}

\usepackage{regexpatch}
\makeatletter
\edef\switcht@albion{%
  \relax\unexpanded\expandafter{\switcht@albion}%
}
\xpatchcmd*{\switcht@albion}{ \def}{\def}{}{}
\xpatchcmd{\switcht@albion}{\relax}{}{}{}
\edef\switcht@deutsch{%
  \relax\unexpanded\expandafter{\switcht@deutsch}%
}
\xpatchcmd*{\switcht@deutsch}{ \def}{\def}{}{}
\xpatchcmd{\switcht@deutsch}{\relax}{}{}{}
\edef\switcht@francais{%
  \relax\unexpanded\expandafter{\switcht@francais}%
}
\xpatchcmd*{\switcht@francais}{ \def}{\def}{}{}
\xpatchcmd{\switcht@francais}{\relax}{}{}{}
\makeatother

\usepackage{ifluatex}
\ifluatex
  \usepackage{fontspec}
  \usepackage[english]{selnolig}
\fi

\iftrue % use default-font
  \ifluatex
    \setmainfont{Latin Modern Roman}
    \setsansfont{Latin Modern Sans}
    \setmonofont{Latin Modern Mono} % "variable=false"
  \else
    \usepackage[%
      rm={oldstyle=false,proportional=true},%
      sf={oldstyle=false,proportional=true},%
      tt={oldstyle=false,proportional=true,variable=false},%
      qt=false%
    ]{cfr-lm}
  \fi
\else
  \ifluatex
    \setmainfont{TeX Gyre Termes}
    \setsansfont[Scale=.9]{TeX Gyre Heros}
    \setmonofont{Latin Modern Mono} % "variable=false"
  \else
    \usepackage{newtxtext}
    \usepackage{newtxmath}
    \usepackage[zerostyle=b,scaled=.9]{newtxtt}
  \fi
\fi

\ifluatex
\else
  \usepackage[T1]{fontenc}
  \usepackage[utf8]{inputenc} %support umlauts in the input
\fi

\usepackage{graphicx}
\usepackage{upquote}
\usepackage{booktabs}
\usepackage{paralist}
\usepackage{csquotes}
\usepackage{textcmds}

% Enable margin kerning
\RequirePackage[%
  babel,%
  final,%
  expansion=alltext,%
  protrusion=alltext-nott]{microtype}%
\DisableLigatures{encoding = T1, family = tt* }

\usepackage{url}
\makeatletter
\g@addto@macro{\UrlBreaks}{\UrlOrds}
\makeatother

% Required for package pdfcomment later
\usepackage{xcolor}

% For listings
\usepackage{listings}
\lstset{%
  basicstyle=\ttfamily,%
  columns=fixed,%
  basewidth=.5em,%
  xleftmargin=0.5cm,%
  captionpos=b}%
\renewcommand{\lstlistingname}{List.}
\usepackage{chngcntr}
\AtBeginDocument{\counterwithout{lstlisting}{section}}

% Enable nice comments
\usepackage{pdfcomment}

\newcommand{\commentontext}[2]{\colorbox{yellow!60}{#1}\pdfcomment[color={0.234 0.867 0.211},hoffset=-6pt,voffset=10pt,opacity=0.5]{#2}}
\newcommand{\commentatside}[1]{\pdfcomment[color={0.045 0.278 0.643},icon=Note]{#1}}

\newcommand{\todo}[1]{\commentatside{#1}}
% Compatiblity with package fixmetodonotes
\newcommand{\TODO}[1]{\commentatside{#1}}

% Bibliography
\ifluatex
\else
  \SetExpansion
  [ context = sloppy,
    stretch = 30,
    shrink = 60,
    step = 5 ]
  { encoding = {OT1,T1,TS1} }
  { }
\fi

% Put footnotes below floats
% Source: https://tex.stackexchange.com/a/32993/9075
\usepackage{stfloats}
\fnbelowfloat

\usepackage{hyperref}
\hypersetup{hidelinks,
  colorlinks=true,
  allcolors=black,
  pdfstartview=Fit,
  breaklinks=true}

% Enable correct jumping to figures when referencing
\usepackage[all]{hypcap}

\usepackage[group-four-digits,per-mode=fraction]{siunitx}

\usepackage[capitalise,nameinlink]{cleveref}

% Nice formats for \cref
\usepackage{iflang}
\IfLanguageName{ngerman}{
  \crefname{table}{Tab.}{Tab.}
  \Crefname{table}{Tabelle}{Tabellen}
  \crefname{figure}{\figurename}{\figurename}
  \Crefname{figure}{Abbildungen}{Abbildungen}
  \crefname{equation}{Gleichung}{Gleichungen}
  \Crefname{equation}{Gleichung}{Gleichungen}
  \crefname{listing}{\lstlistingname}{\lstlistingname}
  \Crefname{listing}{Listing}{Listings}
  \crefname{section}{Abschnitt}{Abschnitte}
  \Crefname{section}{Abschnitt}{Abschnitte}
  \crefname{paragraph}{Abschnitt}{Abschnitte}
  \Crefname{paragraph}{Abschnitt}{Abschnitte}
  \crefname{subparagraph}{Abschnitt}{Abschnitte}
  \Crefname{subparagraph}{Abschnitt}{Abschnitte}
}{
  \crefname{section}{Sect.}{Sect.}
  \Crefname{section}{Section}{Sections}
  \crefname{listing}{\lstlistingname}{\lstlistingname}
  \Crefname{listing}{Listing}{Listings}
}

% Solution for hyperlink refs.
\newcommand{\Vlabel}[1]{\label[line]{#1}\hypertarget{#1}{}}
\newcommand{\lref}[1]{\hyperlink{#1}{\FancyVerbLineautorefname~\ref*{#1}}}
    
\usepackage{xspace}
%\newcommand{\eg}{e.\,g.\xspace}
%\newcommand{\ie}{i.\,e.\xspace}
\newcommand{\eg}{e.\,g.,\ }
\newcommand{\ie}{i.\,e.,\ }

% Powerset
\DeclareFontFamily{U}{MnSymbolC}{}
\DeclareSymbolFont{MnSyC}{U}{MnSymbolC}{m}{n}
\DeclareFontShape{U}{MnSymbolC}{m}{n}{
  <-6>    MnSymbolC5
  <6-7>   MnSymbolC6
  <7-8>   MnSymbolC7
  <8-9>   MnSymbolC8
  <9-10>  MnSymbolC9
  <10-12> MnSymbolC10
  <12->   MnSymbolC12%
}{}
\DeclareMathSymbol{\powerset}{\mathord}{MnSyC}{180}

% Name says it all.
\ifluatex
\else
  \input glyphtounicode
  \pdfgentounicode=1
\fi

% Correct bad hypenation.
\hyphenation{op-tical net-works semi-conduc-tor}

% Some useful info to add.

\iffalse
  \usepackage[intended]{llncsconf}
  \conference{name of the conference}
  \llncs{book editors and title}{0042} %% 0042 is the start page
\fi

% For demonstration purposes only
\usepackage[math]{blindtext}
\usepackage{mwe}
\usepackage[backend=biber, style=numeric]{biblatex}
\addbibresource{java.bib}
\usepackage[ampersand]{easylist}

%%%%%%%%%%%%%%%%%%%%%%%%%%%%%%%%%%%%%%%%%%%%%%%%%%%%%%%%%%%%%%%%%%%%%%%%%%%%%%%%

\title{ec\textbf{\o}}
\author{
    Pablo Acereda Gracia \and
    David Emanuel Craciunesuc \and
    Pablo Martínes García \and
    Laura Pérez Medeiro
}

\date{October 2019}

\begin{document}

\maketitle

%%%%%%%%%%%%%%%%%%%%%%%%%%%%%%%%%%%%%%%%%%%%%%%%%%%%%%%%%%%%%%%%%%%%%%%%%%%%%%%%

\section*{Introduction}

This project will consist of an intelligent system with capabilities to control
abiotic factors that affect the wellbeing of plants, with the objective of
improviing their health and quality of life.

Factors such as humidity and temperature, as well as the levels of light these
receive, will be monitorized and analyzed frequently in order to verify and
ensure optimal quality of life for the plants.

Thanks to these measurements, a series of alarms and different warning states
will be implemented in order to better control the status of the plants and
alert the users of their current situation.

Users will be able to easily install the system themselves, and interact and
control it via a virtual assistant such as \textit{Amazon Alexa}.

%%%%%%%%%%%%%%%%%%%%%%%%%%%%%%%%%%%%%%%%%%%%%%%%%%%%%%%%%%%%%%%%%%%%%%%%%%%%%%%%

\section*{Context}

    %%%%%%%%%%%%%%%%%%%%%%%%%%%%%%%%%%%%

    \subsection*{Current situation of the presented problem}

    The market has seen its fair share of intelligent systems created to aid
    with plant irrigation control, and there are those that even come with a
    mobile app interface, such as Blossom, PlantLink or Edyn. Some of the
    newcomers to jump aboard the backyard-sprinkler train are the so-called
    `intelligent flowerpots', like the one \textit{Xiaomi} has recently started
    to sell, capable of watering the plants automatically depending on the
    humidity of the soil.

    There are other projects like \textit{GR0} that are capable of suggesting
    what plant species one should buy by analyzing the quality and type of soil
    one uses.

    Nevertheless, none of these offers use any kind of virtual-assistant
    integration or any well-designed user experience, for that matter.
    \textit{That} will be the main difference our project will have. Not only
    will the user control the system through a virtual assistant, the design and
    user experience is planned to be exceptional and extremely easy and
    intuitive.

    %%%%%%%%%%%%%%%%%%%%%%%%%%%%%%%%%%%%

    \subsection*{End-of-project prediction}

    Once the course finishes, our intention is to have created a device capable
    of auomatically keeping alive crops or plants with interactions through
    \textit{Amazon Alexa}.

    In order to achieve that, our device will be able to give accurate weather
    predictions, alert of possible plagues that might attack the crops and
    monitor the optimal conditions for the plants themselves.

    %%%%%%%%%%%%%%%%%%%%%%%%%%%%%%%%%%%%

    \subsection*{Target audience}

    This project aims to aid the gardening aficionados that do not have a great
    amount of time at their disposal to take care of their plants optimally. It
    also aims to assist farmers that need help with the care of their crops and
    seek to minimize the effect of unexpected and external factors to their
    produce.

%%%%%%%%%%%%%%%%%%%%%%%%%%%%%%%%%%%%%%%%%%%%%%%%%%%%%%%%%%%%%%%%%%%%%%%%%%%%%%%%

\section*{Project Scope}

%%%%%%%%%%%%%%%%%%%%%%%%%%%%%%%%%%%%%%%%%%%%%%%%%%%%%%%%%%%%%%%%%%%%%%%%%%%%%%%%

\section*{Discarded Ideas}

%%%%%%%%%%%%%%%%%%%%%%%%%%%%%%%%%%%%%%%%%%%%%%%%%%%%%%%%%%%%%%%%%%%%%%%%%%%%%%%%

\section*{Technology to Use}

The decisions about the technology that the project will use have taken into
account the experience of the different members that make up the development
area of the group, as well as the actual monetary cost of the different hardware
elements.

Here are some options considered for the project:

%%%%%%%%%%%%%%%%%%%%%%%%%%%%%%%%%%%%%%%%

\subsection*{Controllers}

\begin{easylist}[itemize]

& Arduino: Offers a wide array of boards, with different specifications and it
comes with a ready-to-use programming environment for the development of
applications and a large community that supports development and improvements.
The main downside to this approach was that every single board would have cost
around \$20.

& Raspberry Pi: Considered an initial option, given the simplicity of its use
and the wide array of languages that are compatible with it. In the end it was
discarded given its enormous size relative to the rest of the elements.

& ESP32: This is the option that was chosen in the end because it's a low-cost
SoC (the price is around \$5), it comes with Bluetooth and it's used in a wide
array of IoT projects. It's also compatible with Arduino, allowing it's users to
benefit from the large amount of resources the Arduino community provides.

\end{easylist}

%%%%%%%%%%%%%%%%%%%%%%%%%%%%%%%%%%%%%%%%

\subsection*{Sensors}

\begin{easylist}[itemize]

& DHT22
& DSB18B20:
& LDR:
& YL-69:

\end{easylist}

%%%%%%%%%%%%%%%%%%%%%%%%%%%%%%%%%%%%%%%%

\subsection*{Programming Languages}

Given the chosen microcontroller (ESP32), the options when it comes to
programming have been:

\begin{easylist}[itemize]

& MicroPython\\
    It consists of a very small and extremely optimized Python interpreter
    designed specifically to run on microcontrollers. It was the initial choice
    for the project. Here are some advantajes and disadvantages to
    \textit{MicroPython}.

    && Advantages \\
        &&& Interactive \textit{RELP} \\
        This means it comes with a program that allows the read of commands to
        evaluate them and print the result on screen without the need to
        compile or load the program, nor to load the program in the
        microcontroller itself.

        &&& Big amount of available libraries \\

        &&& Extensibility \\
        It's able to mix and combine code from different sources that require a
        faster execution to a much lower level because of the extension of its
        functions.

    && Disadvantages \\
        

        

& RTOS:

& Mongoose OS:

& Arduino IDE:

\end{easylist}

%%%%%%%%%%%%%%%%%%%%%%%%%%%%%%%%%%%%%%%%%%%%%%%%%%%%%%%%%%%%%%%%%%%%%%%%%%%%%%%%

\section*{Software Development Methodologies}

%%%%%%%%%%%%%%%%%%%%%%%%%%%%%%%%%%%%%%%%%%%%%%%%%%%%%%%%%%%%%%%%%%%%%%%%%%%%%%%%

\section*{Application Architecture}

%%%%%%%%%%%%%%%%%%%%%%%%%%%%%%%%%%%%%%%%%%%%%%%%%%%%%%%%%%%%%%%%%%%%%%%%%%%%%%%%

\section*{Business Model}

%%%%%%%%%%%%%%%%%%%%%%%%%%%%%%%%%%%%%%%%%%%%%%%%%%%%%%%%%%%%%%%%%%%%%%%%%%%%%%%%

\section*{Development Plan}

%%%%%%%%%%%%%%%%%%%%%%%%%%%%%%%%%%%%%%%%%%%%%%%%%%%%%%%%%%%%%%%%%%%%%%%%%%%%%%%%

\section*{Risk Assessment}

%%%%%%%%%%%%%%%%%%%%%%%%%%%%%%%%%%%%%%%%%%%%%%%%%%%%%%%%%%%%%%%%%%%%%%%%%%%%%%%%

\section*{Contingency Plan}

%%%%%%%%%%%%%%%%%%%%%%%%%%%%%%%%%%%%%%%%%%%%%%%%%%%%%%%%%%%%%%%%%%%%%%%%%%%%%%%%

\section*{Overview of the Project}

%%%%%%%%%%%%%%%%%%%%%%%%%%%%%%%%%%%%%%%%%%%%%%%%%%%%%%%%%%%%%%%%%%%%%%%%%%%%%%%%

\printbibliography

\end{document}
