
\section{Tecnología a utilizar}

\subsection*{Hardware}
\subsubsection*{Controladoras}

\subsubsection*{Sensores}
\begin{itemize}
    \item DHT22: Consiste en un sensor analógico para medir valores de humedad y temperatura cuya señal de salida es digital.  Se eligió este modelo debido a su tamaño, factor bastante importante para nuestro proyecto, su precisión y el precio del mismo. \newline Este sensor será utilizado para medir la temperatura y humedad ambiental.
    \item LDR: Es un fotoresistor, con el cual mediremos la cantidad de luz recibida por la planta.
    \item YL-69: Sensor de humedad de la tierra, nos servirá como activador del sistema de riego o lanzador de notificación para que el usuario conozca que su planta necesita ser regada.
\end{itemize}

\subsection*{Lenguaje de programación}

\begin{itemize}
    
    \item IDE Arduino:  Software de código abierto que cuenta con una amplia comunidad de respaldo. Los lenguajes que admite son C y C++ haciendo uso de reglas especiales de estructuración de código, además de que bajo este IDE se suministran multitud de bibliotecas para procedimientos comunes de entrada y salida. Finalmente ha sido la opción elegida para la programación de la ESP32, ya que la forma de empezar a usar este software con la placa es muy sencilla.
    
\end{itemize}

\section{Metodología de desarrollo}
La metodología escogida para la realización del proyecto ha sido Lean más Kanban con ligeras modificaciones. Los motivos que impulsaron la elección de esta metodología fueron los principios que sigue:

\begin{enumerate}
\item Eliminación de desperdicios. Todo aquello que no aporte valor debe eliminarse (códigos basura, requisitos modificados...)
\item Ampliación del aprendizaje. Aunque inicialmente cada miembro del equipo se encargará del desarrollo de una parte del proyecto, todo el equipo está abierto a ofrecer su ayuda a los demás y a aprender nuevas tecnologías o formas de resolver problemas.
\item Realizar las tomas de decisiones a medida que se van teniendo en cuenta los requisitos a cumplir, así como la manera de enfocarlos. El proyecto ha ido construyéndose de manera progresiva, no se ha seguido una idea fija desde el primer momento sino que esta se ha ido desarrollando y madurando con el tiempo.
\item Entregas rápidas. Cada vez que se realiza una entrega esta implica un aumento de la funcionalidad o una corrección de esta.
\item Potenciar el equipo, todos los miembros han sido partícipes en la toma de decisiones importantes para el proyecto.
\item Creación de integridad, haciendo uso de un sistema para el control de versiones (github en nuestro caso).
\item Visualizar todo en conjunto. \textit{“Pensar en grande, actuar en pequeño, equivocarse rápido y aprender con rapidez”} Esa frase resume la mentalidad de todos los integrantes del equipo. \newline
\end{enumerate}
Esta metodología se ha apoyado con Kanban, el cual no es una metodología de trabajo en sí sino una forma de visualizar el trabajo, controlar la asignación de tareas y mejorar la comunicación de los miembros del equipo.

Lean Kanban es una metodología seguida por varías startups, como IMVU startup creada por Eric Ries quien creó una metodología llamada Lean Startup que se basa principalmente en Lean con Kanban pero con ciertos aspectos más centralizados en el funcionamiento de una startup.

Los ciclos del desarrollo comienzan con el lanzamiento de una idea, la cuál es desarrollada y valorada, se obtiene un aprendizaje con los datos arrojados por esta idea y se incorpora al producto. De esta forma se inició el proyecto con la intención de crear un sistema de detección de riesgos potenciales para los cultivos (clima y plagas principalmente) y se pensó en incluir alexa en el control de este sistema.


\subsection{Aplicación de la metodología}
Para aplicar la metodología de manera eficaz se realizaba semanalmente una breve exposición de los avances realizados (mejoras realizadas, problemas encontrados...), así como un nuevo reparto de tareas y la exposición de alguna nueva idea aplicable al proyecto.

Todo la actividad quedaba registrada en github mediante los commit y el dashboard con las distintas actividades.

\section{Arquitectura de la aplicación}
DIAGRAMA DE PABLO +  EXPLICACIÓN DE ESTE

\section{Modelo de negocio}

MODELO CANVAS AQUÍ 

\section{Planificación temporal y de desarrollo}
La planificación se ha ido elaborando semanalmente, según se tenía mayor conocimiento de las labores a realizar. Estas tareas y el encargado de realizarlas se refleja en el dashboard que nos ofrece github.
En líneas generales, para noviembre se contaría con toda la información relativa a los sensores a utilizar, el esquema de las conexiones a realizar y un servidor donde almacenar los datos de los sensores. Así mismo, se avanzaría con la interfaz de voz de Alexa mediante la realización de un curso para tener un mayor conocimiento acerca de su funcionamiento y la manera de integrarlo en el proyecto.

\section{Riesgos del proyecto}



Cosos de Pablo de seguridad

\section{Plan de contingencias}


\section{Resumen y conclusiones}
La Antorcha Humana fue a bankia y le denegaron una hipoteca.

\end{document}


