%%%%%%%%%%%%%%%%%%%%%%%%%%%%%%%%%%%%%%%%%%%%%%%%%%%%%%%%%%%%%%%%%%%%%%%%%%%%%%%%
% Ubiquitous Computation -                                                     %
%   - Pablo Acereda García                                                     %
%   - David Emanuel Craciunescu                                                %
%   - Laura Pérez Medeiro                                                      %
%%%%%%%%%%%%%%%%%%%%%%%%%%%%%%%%%%%%%%%%%%%%%%%%%%%%%%%%%%%%%%%%%%%%%%%%%%%%%%%%

%%%%%%%%%%%%%%%%%%%%%%%%%%%%%%%%%%%%%%%%%%%%%%%%%%%%%%%%%%%%%%%%%%%%%%%%%%%%%%%%
%                                                                              %
%                               LaTeX Formatting                               %
%                                                                              %
%%%%%%%%%%%%%%%%%%%%%%%%%%%%%%%%%%%%%%%%%%%%%%%%%%%%%%%%%%%%%%%%%%%%%%%%%%%%%%%%

% !TeX spellcheck   = en-US
% !TeX encoding     = utf8
% !TeX program      = pdflatex
% !BIB program      = bibtex
% -*- coding:utf-8 mod:LaTeX -*-

% "a4paper" enables:
%  - easy print out on DIN A4 paper size

% English documents: pass english as class option
\documentclass[english,runningheads,a4paper]{llncs}[2018/03/10]
\usepackage[ngerman,main=english]{babel}
\addto\extrasenglish{\languageshorthands{ngerman}\useshorthands{"}}

\usepackage{regexpatch}
\makeatletter
\edef\switcht@albion{%
  \relax\unexpanded\expandafter{\switcht@albion}%
}
\xpatchcmd*{\switcht@albion}{ \def}{\def}{}{}
\xpatchcmd{\switcht@albion}{\relax}{}{}{}
\edef\switcht@deutsch{%
  \relax\unexpanded\expandafter{\switcht@deutsch}%
}
\xpatchcmd*{\switcht@deutsch}{ \def}{\def}{}{}
\xpatchcmd{\switcht@deutsch}{\relax}{}{}{}
\edef\switcht@francais{%
  \relax\unexpanded\expandafter{\switcht@francais}%
}
\xpatchcmd*{\switcht@francais}{ \def}{\def}{}{}
\xpatchcmd{\switcht@francais}{\relax}{}{}{}
\makeatother

\usepackage{ifluatex}
\ifluatex
  \usepackage{fontspec}
  \usepackage[english]{selnolig}
\fi

\iftrue % use default-font
  \ifluatex
    \setmainfont{Latin Modern Roman}
    \setsansfont{Latin Modern Sans}
    \setmonofont{Latin Modern Mono} % "variable=false"
  \else
    \usepackage[%
      rm={oldstyle=false,proportional=true},%
      sf={oldstyle=false,proportional=true},%
      tt={oldstyle=false,proportional=true,variable=false},%
      qt=false%
    ]{cfr-lm}
  \fi
\else
  \ifluatex
    \setmainfont{TeX Gyre Termes}
    \setsansfont[Scale=.9]{TeX Gyre Heros}
    \setmonofont{Latin Modern Mono} % "variable=false"
  \else
    \usepackage{newtxtext}
    \usepackage{newtxmath}
    \usepackage[zerostyle=b,scaled=.9]{newtxtt}
  \fi
\fi

\ifluatex
\else
  \usepackage[T1]{fontenc}
  \usepackage[utf8]{inputenc} %support umlauts in the input
\fi

\usepackage{graphicx}
\usepackage{upquote}
\usepackage{booktabs}
\usepackage{paralist}
\usepackage{csquotes}
\usepackage{textcmds}

% Enable margin kerning
\RequirePackage[%
  babel,%
  final,%
  expansion=alltext,%
  protrusion=alltext-nott]{microtype}%
\DisableLigatures{encoding = T1, family = tt* }

\usepackage{url}
\makeatletter
\g@addto@macro{\UrlBreaks}{\UrlOrds}
\makeatother

% Required for package pdfcomment later
\usepackage{xcolor}

% For listings
\usepackage{listings}
\lstset{%
  basicstyle=\ttfamily,%
  columns=fixed,%
  basewidth=.5em,%
  xleftmargin=0.5cm,%
  captionpos=b}%
\renewcommand{\lstlistingname}{List.}
\usepackage{chngcntr}
\AtBeginDocument{\counterwithout{lstlisting}{section}}

% Enable nice comments
\usepackage{pdfcomment}

\newcommand{\commentontext}[2]{\colorbox{yellow!60}{#1}\pdfcomment[color={0.234 0.867 0.211},hoffset=-6pt,voffset=10pt,opacity=0.5]{#2}}
\newcommand{\commentatside}[1]{\pdfcomment[color={0.045 0.278 0.643},icon=Note]{#1}}

\newcommand{\todo}[1]{\commentatside{#1}}
% Compatiblity with package fixmetodonotes
\newcommand{\TODO}[1]{\commentatside{#1}}

% Bibliography
\ifluatex
\else
  \SetExpansion
  [ context = sloppy,
    stretch = 30,
    shrink = 60,
    step = 5 ]
  { encoding = {OT1,T1,TS1} }
  { }
\fi

% Put footnotes below floats
% Source: https://tex.stackexchange.com/a/32993/9075
\usepackage{stfloats}
\fnbelowfloat

\usepackage{hyperref}
\hypersetup{hidelinks,
  colorlinks=true,
  allcolors=black,
  pdfstartview=Fit,
  breaklinks=true}

% Enable correct jumping to figures when referencing
\usepackage[all]{hypcap}

\usepackage[group-four-digits,per-mode=fraction]{siunitx}

\usepackage[capitalise,nameinlink]{cleveref}

% Nice formats for \cref
\usepackage{iflang}
\IfLanguageName{ngerman}{
  \crefname{table}{Tab.}{Tab.}
  \Crefname{table}{Tabelle}{Tabellen}
  \crefname{figure}{\figurename}{\figurename}
  \Crefname{figure}{Abbildungen}{Abbildungen}
  \crefname{equation}{Gleichung}{Gleichungen}
  \Crefname{equation}{Gleichung}{Gleichungen}
  \crefname{listing}{\lstlistingname}{\lstlistingname}
  \Crefname{listing}{Listing}{Listings}
  \crefname{section}{Abschnitt}{Abschnitte}
  \Crefname{section}{Abschnitt}{Abschnitte}
  \crefname{paragraph}{Abschnitt}{Abschnitte}
  \Crefname{paragraph}{Abschnitt}{Abschnitte}
  \crefname{subparagraph}{Abschnitt}{Abschnitte}
  \Crefname{subparagraph}{Abschnitt}{Abschnitte}
}{
  \crefname{section}{Sect.}{Sect.}
  \Crefname{section}{Section}{Sections}
  \crefname{listing}{\lstlistingname}{\lstlistingname}
  \Crefname{listing}{Listing}{Listings}
}

% Solution for hyperlink refs.
\newcommand{\Vlabel}[1]{\label[line]{#1}\hypertarget{#1}{}}
\newcommand{\lref}[1]{\hyperlink{#1}{\FancyVerbLineautorefname~\ref*{#1}}}
    
\usepackage{xspace}
%\newcommand{\eg}{e.\,g.\xspace}
%\newcommand{\ie}{i.\,e.\xspace}
\newcommand{\eg}{e.\,g.,\ }
\newcommand{\ie}{i.\,e.,\ }

% Powerset
\DeclareFontFamily{U}{MnSymbolC}{}
\DeclareSymbolFont{MnSyC}{U}{MnSymbolC}{m}{n}
\DeclareFontShape{U}{MnSymbolC}{m}{n}{
  <-6>    MnSymbolC5
  <6-7>   MnSymbolC6
  <7-8>   MnSymbolC7
  <8-9>   MnSymbolC8
  <9-10>  MnSymbolC9
  <10-12> MnSymbolC10
  <12->   MnSymbolC12%
}{}
\DeclareMathSymbol{\powerset}{\mathord}{MnSyC}{180}

% Name says it all.
\ifluatex
\else
  \input glyphtounicode
  \pdfgentounicode=1
\fi

% Correct bad hypenation.
\hyphenation{op-tical net-works semi-conduc-tor}

% Some useful info to add.

\iffalse
  \usepackage[intended]{llncsconf}
  \conference{name of the conference}
  \llncs{book editors and title}{0042} %% 0042 is the start page
\fi

% For demonstration purposes only
\usepackage[math]{blindtext}
\usepackage{mwe}
\usepackage[backend=biber, style=numeric]{biblatex}
\addbibresource{java.bib}
\usepackage[ampersand]{easylist}

%%%%%%%%%%%%%%%%%%%%%%%%%%%%%%%%%%%%%%%%%%%%%%%%%%%%%%%%%%%%%%%%%%%%%%%%%%%%%%%%
%                                                                              %
%                               Actual content                                 %
%                                                                              %
%%%%%%%%%%%%%%%%%%%%%%%%%%%%%%%%%%%%%%%%%%%%%%%%%%%%%%%%%%%%%%%%%%%%%%%%%%%%%%%%

\title{ec\textbf{\o}, a Rainforest Product}
\author{
    Pablo Acereda García \and
    David E. Craciunescu \and
    Laura Pérez Medeiro
}

\begin{document}

\maketitle

%%%%%%%%%%%%%%%%%%%%%%%%%%%%%%%%%%%%%%%%%%%%%%%%%%%%%%%%%%%%%%%%%%%%%%%%%%%%%%%%

\section*{Introduction}

The `\textit{ec\textbf{\o}}` project consists of an intelligent system with the
capability to monitor and control the wellbeing of plants, with the objective of
maintaining and even improving their health.

The way it works is straigtforward, place a sensor package\footnote{Commercially
called a \textit{pod}.} next to the plant(s) to be monitored, and freely connect
it to the suite of plant analysis and control tools \textit{Rainforrest} proudly
provides for free.

Factors such as humidity, temperature, light levels, etc. \ will be monitorized
and analyzed frequently in order to ensure optimal quality of life for the
plants.

The users will be informed of the status of their plants on demand. They will be
able to set up scheduled updates, warnings, and even receive recommendations,
all based on the collected data, and on guidelines set by the users themselves.
They will also be able to remotely perform actions like watering their plants,
checking the pH levels of the soil, or even detecting if the plants are
suffering from various illnesses\footnote{Features in development and not
representative of final product.}.

The key differentiating factor of the project in comparison to other market
choices is the ability to let the user \textit{forget} without any consequences.
They are in total control and their interactions with the system are made as
simple as possible, going as far as to even providing integration with virtual
assistants such as \textit{Amazon Alexa}.

%%%%%%%%%%%%%%%%%%%%%%%%%%%%%%%%%%%%%%%%%%%%%%%%%%%%%%%%%%%%%%%%%%%%%%%%%%%%%%%%

\section*{Context}

    %%%%%%%%%%%%%%%%%%%%%%%%%%%%%%%%%%%%

    \subsection*{Current situation of the presented problem}

    The market has seen its fair share of intelligent systems created to aid
    with plant irrigation control, and there are those that even come with a
    mobile app interface, such as Blossom, PlantLink or Edyn. Some of the
    newcomers to jump aboard the backyard-sprinkler train are the so-called
    `intelligent flowerpots', like the one \textit{Xiaomi} has recently started
    to sell, capable of watering the plants automatically depending on the
    humidity of the soil.

    There are other projects like \textit{GR0} that are capable of suggesting
    what plant species one should buy by analyzing the quality and type of soil
    one uses. And there are even those that aim to turn one's plant into a pet,
    like \textit{Lua}, so that not taking care of it feels worse than simply
    forgetting to water the plant.

    Nevertheless, most of these products weren't successful or don't even exist
    anymore. Their way to market, their price when compared to their value, or
    even their actual need to the average consumer were all lacking. The
    difference \textit{Rainforrest} offers is very simple:
    \textit{ec\textbf{\o}} isn't about turning one's plant into a pet or buying
    a flowerpot that just becomes another item to remember about. It is about
    finally having the ability to simply let plants take care of themselves
    without having to worry or even remember they are there. It is not a system
    designed to make the users remember, but to (whenever possible) let them
    forget without any consequences.

    %%%%%%%%%%%%%%%%%%%%%%%%%%%%%%%%%%%%

    \subsection*{End-of-project prediction}

    The end-of-project predictions are \textit{extremely simple}. Once the
    project is finished, our intention is to have created a device and suite of
    functionality around it capable of automatically keeping most indoor plants
    alive, with as little human interaction as possible. All the while keeping
    the users informed as they see fit, and providing them with all the tools
    necessary for the care of their plants.

    %%%%%%%%%%%%%%%%%%%%%%%%%%%%%%%%%%%%

    \subsection*{Target audience}

    This project aims to aid the plant enthusiast that does not have a great
    amount of time at their disposal to take care of their plants. It also aims
    at all the entities, businessess, spaces, etc.\ where plants can be found, 
    but their care is often forgotten about. More generally, the target audience
    is anyone or anything that is involved with the care of indoor plants in any
    way.

%%%%%%%%%%%%%%%%%%%%%%%%%%%%%%%%%%%%%%%%%%%%%%%%%%%%%%%%%%%%%%%%%%%%%%%%%%%%%%%%

\section*{Project Scope}

The main objective of the project at hand is the creation of a prototype to
present to multiple potential investors in order to seek and obtain further
funding. The `\textit{ec\textbf{\o}}' system is currently the only product of
the startup \textit{Rainforrest}, therefore this project is of vital importance
to the company.

`\textit{ec\textbf{\o}}' can be described both as the physical set of sensors
(commercially known as \textit{pods}) and the suite of software and utilities
provided by \textit{Rainforrest} for their control and interaction. The suite is
free to use, but the only way to make true use of it is through the purchase of
a certified \textit{Rainforrest} `\textit{ec\textbf{\o}}' \textit{pod}.

The project has the potential for a lot more. Due to the position in which it is
presented and the objective \textit{Rainforrest} hopes to achieve with it, it
has been stripped down to its most basic functionalities, so that once presented
and sucessfully funded, the rest can be built upon a product that is already
viable thanks only to its core functions.

%%%%%%%%%%%%%%%%%%%%%%%%%%%%%%%%%%%%%%%%%%%%%%%%%%%%%%%%%%%%%%%%%%%%%%%%%%%%%%%%

\section*{Discarded Ideas}

As with any project, the final result usually ends up being pretty different
from the initial one. The following list contains some of the discarded ideas 
for the creation of `\textit{ec\textbf{\o}}':

\begin{easylist}[itemize]

& \textbf{Development of own microcontroller} \\
This would have been extremely useful for the product itself, given that the
development of a microcontroller specialized for the task would have optimized
the way in which the product was designed. After much thought, this idea was
discarded, given the short timeframe of the product development and the fact
that this is still yet a prototype.

& \textbf{Detection of plagues} \\
One of the initial ideas was that the system should be able to detect different
insects and monitor for known plagues depending on time of year, soil, plant,
geolocation, etc. \ Some initial ideas went as far as to consider the use of
infrared radiation in order to provide the project with this feature. In the
end, this idea was discarded given the difficulty of its implementation.

& \textbf{Adaptation for crops and large-scale usage} \\
Given the prototypic nature of the project, and the ambition of such an idea,
this feature has yet to be explored.

\end{easylist}

%%%%%%%%%%%%%%%%%%%%%%%%%%%%%%%%%%%%%%%%%%%%%%%%%%%%%%%%%%%%%%%%%%%%%%%%%%%%%%%%

\section*{Technology to Use}

This section contains the different options that were considered for the
project. The different decisions about the technologies that were to be used all
took into account the experience of the various members of the group, as well as
the monetary cost of the different hardware elements, their general availability
, and the trust the manufacturer itself inspired.

%%%%%%%%%%%%%%%%%%%%%%%%%%%%%%%%%%%%%%%%

    \subsection*{Controllers}

    \begin{easylist}[itemize]

    & \textbf{Arduino}

    \textit{Arduino} is an open-source electronics platform that is based on
    easy-to-use software and hardware. The \textit{Arduino} boards themselves 
    are controlled by sending a set of instructions to a microcontroller on the 
    board. To do so, one must use the \textit{Arduino} programming language, 
    which is based on \textit{Wiring}, and the \textbf{Arduino Software (IDE)}, 
    which is based on \textit{Processing}.

    These boards were the first choice because they are extremely easy to use. 
    The programming language is extremely easy to pick up if one already knows
    \textit{C}, and the \textit{Arduino} community offers a wide array of
    free-to-use resources that improve the quality and reduce the effort of any 
    project trying to use \textit{Arduino}.

    Even with all its benefits, \textit{Arduino} ended up being discarded as a
    possible option, given that the absolutely cheapest of boards would still 
    cost around \$20.

    & \textbf{Raspberry Pi}

    According to the official \textit{Raspberry Pi} webpage: \textit{``Raspberry
    Pi is a low cost, credit-card sized computer that plugs into a monitor or a 
    TV, and uses standard keyboard and mouse. (\ldots) It's capable of doing 
    what you'd normally expect a regular computer to do, from browsing the 
    internet and playing high-definition video, to making spreadsheets, word 
    processing and playing games''}.

    Just like a regular computer, one would be able to program it to do 
    whatever they'd want it to do. This would have been ideal for the project 
    itself, given the simplicity of its use and the gigantic array of languages 
    that are compatible with it. In the end, though, it was discarded because 
    the computer itself was too large in comparison to the rest of the elements 
    of the project.

    & \textbf{ESP32}

    The \textit{ESP32} is a series of low-cost, low-power system on a chip
    \textbf{SoC} microcontrollers with integrated Wi-Fi and dual-mode Bluetooth.
    This was the option the team ended up choosing for the project, given that 
    the \textit{ESP32} was specifically designed for \textbf{wearable 
    electronics and IoT applications.}

    One can also program on it using the \textit{Arduino IDE}, which was a huge
    advantage. Most of the team already knew how to use and control
    \textit{Arduino}, so not having to learn a skill specifically for the
    microcontroller plus all the benefits of the \textit{Arduino} community made
    the group decide on the \textit{ESP32} as an option.

    \end{easylist}

%%%%%%%%%%%%%%%%%%%%%%%%%%%%%%%%%%%%%%%%

    \subsection*{Sensors}

    There are many different kinds of sensors on the market. Many of these are
    extremely capable and can be obtained very easily. When deciding on the 
    sensors, the team always looked for the most reliable and available ones.
    Here are some of the sensors the project will use:

    \begin{easylist}[itemize]

    & \textbf{DHT22} \\
    It is a basic and low-cost digital temperature and humidity sensor. It 
    measures the surrounding air thanks to a capacity humiditor sensor and a
    thermistor, and spits out digital signal on the data pin. It's extremely 
    simple to use, given that its transmissions are already digital, but one can
    only make readings each two seconds. Extreme speed, or lack thereof, is not
    a highly important factor when it comes to plant humidity and temperature
    measuring, so the only real downside one could see to this sensor turns out
    not to be a downside at all.

    & \textbf{LDR} \\
    A \textit{Light Dependent Resistor} or a photo resistor is a device whose
    resistivity is a function of the incident electromagnetic radiation. These
    are light sensitive devices also called photo conductors, photo conductive
    cells or simply photocells.

    The ability to detect light is a must in the soon-to-be deliverable product,
    therefore the use of this sensor was an easy and obvious solution.

    & \textbf{YL-69} \\
    According to the product description the manufacturer provides:
    \textit{``The YL-69 is a simple sensor that can be used to detect soil
    moisture/relative humidity within the soil. The module is able to detect
    when the soil is too dry or wet. Great for use with automatic plant watering
    systems''}.

    The need for a sensor like this is evident, and what better choice than one
    whose price in comparison to that of the rest of the setup and components is
    practically insignificant. This will be the main triggering mechanism to
    notify that the plant or plants need to be watered soon.

    \end{easylist}

%%%%%%%%%%%%%%%%%%%%%%%%%%%%%%%%%%%%%%%%

    \subsection*{Programming Languages}

    As previously mentioned, the chosen microcontroller is the \textit{ESP32}.
    Therefore, the options when it comes to programming languages have been:

    \begin{easylist}[itemize]
    
    & MicroPython

    \textit{MicroPython} an open source \textit{Python} programming language
    interpreter that is capable of running on small embedded development boards.
    With the adaptability build into \textit{MicroPython}, one can write clean
    and simple \textit{Python} code to control hardware directly instead of
    having to use complex low-level languages.

    Even if a reduced version, \textit{MicroPython} still supports most of
    \textit{Python}'s syntax and implements most of its inner mechanisms. Given
    all these features and advantages, it was a quick initial choice for the
    project.

    These are some of the features that set it apart from other embedded
    systems:

        && \textbf{Interactive REPL}
    This feature allows execution of code without the need of any compilation or
    uploading time, which is perfect for systems with a high level of
    experimentation.

        && \textbf{Extensive software library}
    \textit{MicroPython} already comes with libraries built in to support common
    tasks, like JSON data parsing, regular expression handling or even network
    socket programming.

        && \textbf{Extensibility}
    Advanced users may be able to mix \textit{MicroPython} with extensible
    low-level C/C++ functions in order to further optimize their code and make
    the execution faster when it really matters.

    \textit{MicroPython} has some very useful features. Unfortunately, it also
    comes with its downsides:

        && Slower code and higher memory needs when compared to C/C++.
        && Complicated microcontroller initialization process.
        && Limited functionality for some key libraries.

    & FreeRTOS

    \textit{FreeRTOS} is a real-time operating system made specifically to run 
    on embedded systems. It is especially good because it natively provides core
    real time scheduling, inter-task communication, timing and synchronization
    primitives. This translates into a more accurately controlled kernel and a
    system that is able to execute tasks exactly when they have to be executed, 
    a \textit{deterministic} system.

    In \textit{FreeRTOS}, applications can be assigned in a static manner while
    objects themselves can be assigned dynamically with different 
    memory-assignment schemes. This system was quickly discarded. Even 
    with the great deal of documentation it provides, the complexity of 
    \textit{FreeRTOS} and the learning curve it would entail made it 
    practically impossible in the provided timeframe.

    & Mongoose OS

    \textit{Mongoose OS} is a firmware development framework specifically 
    designed for development of IoT products. It is highly compatible with a 
    wide array of microcontrollers, just like the \textit{ESP32}, and its
    objective is to fill `the noticeable gap for embedded software developers' 
    between firmware created for prototyping and bare-metal microcontrollers.

    It comes with an integrated web server and supports interaction with both
    private and public clouds like AWS IoT or Mosquitto.

    & Arduino IDE

    It's an open-source project that has a very large community behind it.
    \textit{Arduino IDE} comes with a text editor for writing code, a message 
    area, a console, and other common functionalities of full-fledged IDEs. The
    programming languages it admits are C and C++, but with some special rules 
    when it comes to structuring the code. The IDE also comes with plenty of 
    built-in libraries for common tasks like robot control or I/O mechanisms.

    This is the option the team ended up choosing, not only for its simplicity 
    and usefulness, but also because it can interact directly with the 
    \textit{ESP32} chip that the project will use as a main component.

    \end{easylist}

%%%%%%%%%%%%%%%%%%%%%%%%%%%%%%%%%%%%%%%%%%%%%%%%%%%%%%%%%%%%%%%%%%%%%%%%%%%%%%%%

\section*{Development Methodology}

One of the key factors of any project, especially of new ones, is the
\textit{Development Methodology}. These methodologies are a means by which a
product can be created in an organized, somewhat predictable, and planned
manner. A \textit{Development Methodology} is nothing more than the process of
planning, creating, testing, and then deploying a project. The different types
of methodologies themselves only alter the way in which those components
interact with each other.

The members of the team working on the project have a lot of experience working
with one another, and feel very comfortable with their peers. The group chose
this methodology because the members realized they inadvertently already used
\textit{Lean with Kanban} in their other project, even before realizing, or
finding out about the existence of the methodology itself. It was an obvious
choice once they found out about this.

Some of the most important characteristics of this methodology are:

\begin{easylist}[enumerate]

& \textbf{Minimizes wasted resources}. Everything that is not absolutely
necessary is swiftly eliminated, be it garbage code, unnecessary meetings or
even obsolete hardware. This helps maintain the project slim and clean. In
essence, it avoids bloating.

& \textbf{Improves the sharing of knowledge}. Even if a member has a
defined function within the team, and their tasks are clearly stated, any other
member can lend a hand to help if they feel the need or have the time. This
keeps the team agile and fresh.

& \textbf{Makes project scope more flexible}. Within \textit{Lean}, decisions
are made as the requirements are being met, as well as when their focus starts
to change. The project starts building itself in a progressive and adaptative
manner because it hasn't followed a strictly defined idea since its inception.
Instead, it has been allowed to grow and morph into what it needed to become.

& \textbf{Keeps the team involved and oriented}. Every single member is a vital
part of the decision-making process, and they have the ability to define their
own work. This methodology works best in teams that are extremely motivated, and
members usually divide the work amongst themselves without the need for actual
meetings and written-down planification. Thanks to this, the project does not
lose sight of the objective. The moto of the team could be described as
\textit{``Think big, act small, err frequently, and fix quickly''}

\end{easylist}

With all these factors, and especially when combining it all with
\textit{Kanban}\footnote{Kanban is a visualization methodology and mechanism,
not an actual defined development process \textit{per se}.}, controlling the
assignment of different tasks and jobs for every team member was extremely easy
and intuitive. The combination is a key component of the new methodology, since
the visibility of \textit{Kanban} increases the effect \textit{Lean} itself has.

The development cycles in \textit{Lean Kanban} start with the creation of an
idea, which is then assessed and developed in order to analyze if its addition
to the overall project would be of any real value. With this process, the worse
case scenario is still obtaining knowledge about the creation and implementation
of that specific idea, even if it does not end up being part of the project.
This is the way some of the most important parts of `\textit{ec\textbf{\o}}'
were created, like the \textit{Amazon Alexa} virtual-assistant integration or
the detection of factors like humidity and light levels.

    %%%%%%%%%%%%%%%%%%%%%%%%%%%%%%%%%%%%

    \subsection*{Implementation of the metodology}
    
    In order to apply the methodology in an effective manner and keep track of
    everything, the team has used the project collaboration tools of
    \textit{GitHub Projects}, which are officially integrated as part of the
    software of University of Alcalá thanks to the \textit{GitHub Campus
    Program}.

    The team met weekly to plan their activities and tasks for the following
    week and kep in touch with the mentioned tools, also able to track changes
    in the status of each task and maintain the visibility of the overall
    project itself.

    All deliverables, modifications, and all the planification have been
    recorded thanks to \textit{GitHub Projects}. A link to the repository will
    be provided with this document. \todo{Once finished we should add the actual
    link here and a couple of graphs detailing the development of the project.
    This should be a relatively easy task}

%%%%%%%%%%%%%%%%%%%%%%%%%%%%%%%%%%%%%%%%%%%%%%%%%%%%%%%%%%%%%%%%%%%%%%%%%%%%%%%%

\section*{Application Architecture}

The device is formed by two components:
\begin{itemize}
\item Data collector. Formed by all the different sensors that can be found.
This component aims to collect and send data to the server. It is the server
where the information is going to be treated.
\item Actuator. Formed by the irrigation system, it can be activated when the
land's humidity is low or using a voice command.
\end{itemize}

All the data collected by the sensors is sent to a server, using WiFi and
MQTT protocol. The reasons why this protocol has been chosen are, on the 
one hand, the fact that it is once of the most used protocol in which IoT is
referred,
and also the fact that we can use OpenMQTTGateway which provided 
solutions to managed different protocols used by different devices (which means
that if there is any sensor that uses BLE, it can be translated into MQTT).
There is where C takes part, as it is the language in which main.cpp will be
created and where all the configuration of MQTT can be found. In the beginning,
data is stored into a relational Database which used PostgreSQL as DBMS. Then,
data is sent to the AWS DataBase which will be used by Alexa's API to give
the user information about how the pant is.

Also, the provided information by the sensors will be used in a web page, where
register users can find some graphics and statistics.

%%%%%%%%%%%%%%%%%%%%%%%%%%%%%%%%%%%%%%%%%%%%%%%%%%%%%%%%%%%%%%%%%%%%%%%%%%%%%%%%

\section*{Business Model}

When it comes to selling and pitching the product to customers, there are many
factors to take into account. Different elements such as the target demographic
or the country in which the product will be commercialized have to be taken into
account. In order to be successful, the business team at \textit{Rainforrest}
has devised a careful and detailed plan in order to optimize the labor of the
company.

There are many things a company must ask itself before selling their products.
The first factor of many is about what the company offers, what value do the
things that the company does have.

%%%%%%%%%%%%%%%%%%%%%%%%%%%%%%%%%%%%%%%%

    \subsection*{Offering. Value Proposition}

    It is assumed that everyone would love to have plants at home without the
    need to worry about them having enough water or receiving enough sunlight.
    Many times, people buy plants to simply decorate their homes, not expecting
    their maintenance to be such a burden. Others may love plants but then claim
    to have no time, ability, or real interest whatsoever when it comes to
    their wellbeing and health.

    `\textit{ec\textbf{\o}}' can assure its customers that their plants will be
    well attended at all times. It constantly measures the state of the plant by
    checking its temperature, humidity, pH levels, etc. After collecting all the
    information needed, it carries out a diagnostic and it lets the owner know
    whether the plant is in optimal conditions or not.

    The product itself adapts to the needs of each plant, providing its
    customers with a simple, time-saving, innovative and personalized solution.
    Forgetful or inexperienced customers need not worry about possibly letting
    their plants die anymore, and plant lovers will have the ability to access
    information about the current state of their plants at all times. Combined
    with the huge, functionalities one has come to expect of the
    \textit{Rainforrest} family of tools, `\textit{ec\textbf{\o}}' provides an
    even more convenient and efficient way of ensuring the wellbeing of the
    customer's plants.

%%%%%%%%%%%%%%%%%%%%%%%%%%%%%%%%%%%%%%%%

    \subsection*{Customers}

    This section will contain information related to the customers themselves,
    the point of view the product should have when presented to them, different
    `customer segments', etc.

        %%%%%%%%%%%%%%%%%%%%%%%%%%%%%%%%%%%%

        \subsubsection*{Customer segments}

        Taking care of elders, children, and animals is considered not only
        normal, but honorable. What about plants? They are a key factor in the
        future of the planet and the lives of human beings, so there's a general
        incentive among the population to take care of them. By buying
        \textit{ec\textbf{\o}} this task will be quicker and easier than ever.

        Assuming that everyone would share this same motivation would be
        foolish. Implemented a segmentation strategy would be recommended in
        order to reach as many potential customers as possible.
    
        The key aspects to follow within the segmentation strategy are focused
        on lifestyle, the income of the potential customer, and
        invidual/business approach. It seems obvious that \textit{ec\textbf{\o}}
        will stick to a niche unattended market formed by people that love
        plants and have them at home as part of their lifestyles, businesses
        whose main activity is related to plants, or enterprises in sectors such
        as horticulture, agriculture or gardening. Another option would be
        targeting public spaces and entities such a governments, public parks,
        botanic gardens or governmental buildings.

        %%%%%%%%%%%%%%%%%%%%%%%%%%%%%%%%%%%%

        \subsubsection*{Channels}
        
        The channels through which a company can deliver its value proposition
        is perhaps one of the most important factors when commercializing a
        product. Reaching each different segment with the adequate communication
        strategy is not only important, but vital to the process. We may find
        multiple possible client groups to which to present the products.
       

        First of all, the company should focus on the business-to-business and
        the government deals, since these normally go hand-in-hand with
        discounts due to a large volume of purchases, and business negotiations
        are at stake.

        When it comes to individual clients, a large range of options are
        available. In order to attract attention, the company should promote
        itself through advertisements on specialized magazines such as `Country
        Gardens', `Fine Gardening', or `Horticulture', which are read by people
        interested in plants, and that would probably be interested in
        \textit{ec\textbf{\o}} as well.

        Another option would be to create online advertisements through
        gardening forums and blogs, social media such as \textit{facebook}
        groups, \textit{instagram} targeted posts, etc. \

        Another promising option would be that of flyers explaining the benefits
        of \textit{ec\textbf{\o}} and its features, to be distributed in botanic
        gardens or plant-specialized stores. Indirect advertisements could be
        very efficient to find \textit{ec\textbf{\o}}'s target customers, and
        word-of-mouth among friends, relatives, and colleagues who have bought
        the product and are satisfied with it.

        \textit{ec\textbf{\o}} will only have an official distribution channel:
        online selling, where buyers can directly buy the product online.

        %%%%%%%%%%%%%%%%%%%%%%%%%%%%%%%%%%%%

        \subsubsection*{Customer Relationships}

        The relatioship between the customer and \textit{Rainforrest} does not
        end once they buy a product. The company will create a phone line and
        email address, so that the customer who are having issues with the
        product, or have information requests can contact the company directly.
        Therefore, personal assistance will be available along the whole
        purchase process, either during pre-sales and/or after sales situations.

        %%%%%%%%%%%%%%%%%%%%%%%%%%%%%%%%%%%%

    \subsection*{Infrastructure}

        \subsubsection*{Key Activities}

        In order to execute \textit{Rainforrest}'s value proposition, the
        following key activities must be executed: (1) design, creation, and
        manufacture of the \textit{pod} system, and (2) development of the
        \textit{Rainforrest} suite of free tools to go with it. Another
        potential activity would be to target the product's potential customers
        correctly by investing appropriately in the different channels needed to
        create awareness in customer's minds that, hopefully, will end up in the
        product's purchase.
    
        %%%%%%%%%%%%%%%%%%%%%%%%%%%%%%%%

        \subsubsection*{Key Resources}

        In order to create \textit{ec\textbf{\o}}, and sustain the company's
        business, the physical components for the development of the
        \textit{pods} are needed, as well as the human capital crucial for the
        creation and development of the \textit{Rainforrest} suite.

        %%%%%%%%%%%%%%%%%%%%%%%%%%%%%%%%

        \subsubsection*{Partner Network}

        Partnerships and alliances are crucial for a business, especially when
        the company is new to the marketplace. They optimize operations and
        reduce risks so that the company can focus on its core activity, which
        in \textit{Rainforest}'s case is the creation of \textit{ec\textbf{\o}}.

        The main and most important partnership for \textit{Rainforest} is the
        one with the physical distributors of their product. Even if the product
        is sold online, it still has to arrive to the client in order for them
        to benefit from it. Choosing the right partners is essential for the
        company to ensure that quality of product is maintained and the
        relationship with the client is kept honest and trustworthy.

        %%%%%%%%%%%%%%%%%%%%%%%%%%%%%%%%

    \subsection*{Finances}

        \subsubsection*{Cost structure}

        \textit{ec\textbf{\o}}'s business structure would be value-driven, since
        the enterprise will focus its efforts on creating value for its
        customers rather than minimizing costs as much as possible. It is clear
        that the company will try to minimize its costs to get higher revenues,
        but this is not the main objective for the firm at the time this
        document was being written.

        In terms of costs, the key activities that will be more expensive
        throughout the process are the technical development and computing
        programming of the project itself, both in terms of the low-level
        software and the multiple user interfaces. Whenever possible, especially
        when selling to businesses and governments, economies of scale could be
        reached, thus, decreasing the cost of each unit as the number of units
        that are ordered of the product increase.

        %%%%%%%%%%%%%%%%%%%%%%%%%%%%%%%%

        \subsubsection*{Revenue Streams}


        Taking a good look at the market, the cost of material to create the
        product would be around \$20US.\ To this cost, other fixed costs such as
        the creation of the \textit{Rainforrest} suite, the design and
        implementation of the \textit{pods} must also be considered. Comparing 
        it to the market, the most similar competitor would be ``Xiaomi Mi Smart
        Flowerpot Plant Pot'' which costs \$50US.\

        The fact that this product is almost unheard of should be more than
        enough to demonstrate that \$50US for a flowepot is universally
        considered to be \textit{too much}, even if it is intelligent and waters
        one's plants automatically.

        This fact provides \textit{Rainforest} with an idea of what price would
        be reasonable to ask for the product and still generate reveue. With
        this, \$35US seems like the maximum anyone would be inclined to pay for
        a product of these characteristics. The revenue streams of the firm will
        mainly come from the sale of this very product and this price would give
        \textit{Rainforest} a decent revenue margin to ensure its success in the
        future.

    %%%%%%%%%%%%%%%%%%%%%%%%%%%%%%%%%%%%

    \subsection*{Marketing}

    When it comes to actually getting to the clients. There are many factors to
    take into account. From the way the company is perceived in the market, to
    the way customers see the product and the sentiment they have towards it. In
    order to maximize the labors of the company and market the product in the
    best way possible, \textit{Rainfores}'s marketing team has devised the
    following factors and elements to follow.

        %%%%%%%%%%%%%%%%%%%%%%%%%%%%%%%%

        \subsubsection*{Perception and differentiation}

        When it comes to \textit{Rainforest} itself, the focus and the angle the
        company should have towards its customers and towards the market could
        be characterized as that of the personality of a young technological
        activist that fights for the future of planet Earth. Some of the most
        important factors this company should have and the way it should behave
        are the following:

        \begin{easylist}[itemize]

        & \textbf{Young}. With \textit{young} the team does not mean the company
        should act childishly, but with the desire to do something and the wish
        to work hard and earn people's trust.

        & \textbf{Connected}. Maintaining a connection with the customer, be it
        through social media or through a different online presence is a key
        factor when appealing to the younger audiences and customers. The very
        inclusion of `technology' in the product will attract younger customers
        towards it and statistics show a company that keeps in touch is a
        company customers trust.

        & \textbf{Revolutionary}. This is not a term that defines the product or
        the way to do things itself. With a \textit{Rainforest} that is
        revolutionary what the marketing team means is a company that is not
        afraid to do things differently and a company that does not simply fear
        change for the sake of change.

        & \textbf{Responsible}. At the same time as young, connected and
        revolutionary, the company should always maintain its status as a
        responsible one. The product \textit{Rainforest} sells is one that helps
        nature and keeps the planet clean. It's a product that helps people
        reduce wasted resources and keep plants alive, therefore one could say
        the product is responsible with the environment, and so should the
        company be.

        & \textbf{Involved}. An involved and present company is one that is sure
        to succeed, especially when it is related to nature and the topics
        \textit{Rainforest} treats and gets involved with. A company that is
        about the betterment of nature and the improvement of it \textit{must}
        be one that is involved.

        & \textbf{Honest}. This point is obvious. A company that does not lie
        and stays honest with their customers is a company that gains their
        trust. A company that inspires security and makes people feel confident
        when buying their products.

        \end{easylist}

%%%%%%%%%%%%%%%%%%%%%%%%%%%%%%%%%%%%%%%%%%%%%%%%%%%%%%%%%%%%%%%%%%%%%%%%%%%%%%%

\section*{Development Plan}

Given the very nature of the software development plan the team has chosen for
the project, the need for an actual `development plan' \textit{per se} is not of
the utmost importance. The team has instead chosen to divide the work in a
realistic set of project objectives and deliverable items within the provided
timeframe of the project.

It is true that projects generally function better with a somewhat defined
development workflow. This case, however, is not applicable to the engineering
team at \textit{Rainforest}. The very nature of their relationship and the way
members are able to collaborate seamlessly given their already based background
makes the \textit{Lean} software development methodology a great one for the
team. In fact, the addition and creation of a software development plan into the
development environment would take more time than the actual development itself.

Every single characteristic found in this document and defined as an objective
can be considered a deliverable item unless specified otherwise. This section is
also accompanied by a document called ``planning'' where a preliminary idea and
conceptualization of how the work should be distributed among the work months
can be shown. This, however is a mere conceptualization of a preliminary idea,
it should not be taken as definitive.

%%%%%%%%%%%%%%%%%%%%%%%%%%%%%%%%%%%%%%%%%%%%%%%%%%%%%%%%%%%%%%%%%%%%%%%%%%%%%%%%

\section*{Risk Assessment and Contingency Plan}

All the information related to this point can be found in the file called as risk\_analysis-contingency\_plan.xlsx

%%%%%%%%%%%%%%%%%%%%%%%%%%%%%%%%%%%%%%%%%%%%%%%%%%%%%%%%%%%%%%%%%%%%%%%%%%%%%%%%

\section*{Overview of the Project}

Even though the project seems to be seriously ambitious, all the team members
are highly motivated and have studied all the relevant information carefully,
with the objective of having the clearest set of mind by the time the project
developing takes place.

Using a close-loop feedback system within the team members, the problems and
difficulties faced through the implementation of the project have been overcome
without difficulty.

\todo{Add some shit when we done about being interested in learning}

%%%%%%%%%%%%%%%%%%%%%%%%%%%%%%%%%%%%%%%%%%%%%%%%%%%%%%%%%%%%%%%%%%%%%%%%%%%%%%%%

\printbibliography

\end{document}
