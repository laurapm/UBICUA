\documentclass[runningheads]{llncs}
% Hipervínculos
\usepackage{hyperref}
% Tablas
\usepackage[table,xcdraw]{xcolor}
% Idioma y tildes
\usepackage[spanish]{babel}
\selectlanguage{spanish}
\usepackage[utf8]{inputenc}

% Listas sencillas
\usepackage[ampersand]{easylist}

\usepackage[utf8]{inputenc}

\title{PECL1. COMPUTACIÓN UBICUA} %inserten aqui el nombre de nuestro proyecto
\author{Pablo Acereda García \and David Emanuel Craciunescu \and Pablo Martínez Gracia \and Laura Pérez Medeiro }
\date{October 2019}


\begin{document}

\maketitle

\section{Introducción}
El proyecto consistirá en un sistema inteligente para controlar los factores abioticos que afectan al bienestar de las plantas, con el objetivo de mejorar la salud de las mismas.

Para ello se monitorizarán la humedad y temperatura tanto exterior como de la tierra así como la cantidad de luz que reciben. Con estos valores y una serie de parámetros se generarán las corresponsientes alarmas que indiquen situaciones de riesgo para la salud de la planta.

\section{Contexto}
    \subsection{Situación actual del problema a abordar}
    Actualmente existen bastantes sistemas que se encargan de controlar el riego de las plantas y los cuales ofrecen la posibilidad de ser controlados mediante dispositivos móviles. Algunos de ellos son Blossom, PlantLink o Edyn. Incluso encontramos maceteros inteligentes, como el que nos ofrece Xiaomi, capaz de regar las plantas automáticamente.También encontramos otros proyectos como GRO, capaz de sugerirnos especies de plantas que podríamos cultivar en función del terreno que disponemos.
    \newline
    Sin embargo, ninguno nos ofrece integración con ningún asistente virtual como pueda ser Alexa, ese será la principal diferencia de nuestro proyecto con los anteriores.
    
    \subsection{Situación prevista al final del proyecto}
    Al finalizar la asignatura se pretende tener un dispositivo capaz de mantener con vida un cultivo de manera automática y el cual pueda ser controlado mediante comandos de voz gracias a Alexa.
    Para ello, nuestro dispositivo contará con la predicción del tiempo, aviso de posibles plagas, estado del terreno y condiciones óptimas necesarias para el tipo de planta que se posea.
    
    \subsection{Beneficiarios del proyecto}
    Con este proyecto se pretende ayudar a los aficionados de la jardinería que no disponen de una gran cantidad de tiempo para el óptimo cuidado de las plantas, así como a agricultores que necesiten una ayuda extra para minimizar el impacto de situaciones climáticas extremas. Por situaciones climáticas extremas se entienden, por ejemplo, heladas producidas en fechas inesperadas.
    
\section{Alcance del proyecto}

\section{Descripción de ideas descartadas}

Inicialmente se pensó en un sistema que fuera capaz de controlar las plagas que atacan a los cultivos, sin embargo, tal proyecto no es sencillo ya que la detección de los insectos no es sencilla. Se pensó en utilizar 

\section{Tecnología a utilizar}
Las decisiones acerca de la tecnología a utilizar en el proyecto se han tomado teniendo en cuenta la experiencia de los distintos miembros que conforman el área de desarrollo, así como el costo que tienen los distintos componentes.
\newline
Las distintas opciones barajadas para la realización del proyecto han sido:

\subsection*{Hardware}
\subsubsection*{Controladoras}
\begin{itemize}
    \item Arduino: Ofrece una amplia gama de placas, con distintas prestaciones, cuenta con un entorno de programación para la creación de aplicaciones y una gran comunidad que apoya el Desarrollo. El principal inconveniente que se encontraba era su precio, el cual ronda los 20 euros.
    
    \item Raspberry: Fue una opción que se estuvo barajando utilizar, ya que ofrece una amplia variedad de lenguajes de programación a utilizar. Sin embargo, fue descartada debido a su gran tamaño.
    
    \item ESP32: Ha sido la opción finalmente elegida, ya que se trata de un SoC de muy bajo coste (su precio ronda los 5 euros), cuenta con Bluetooth y es utilizado en multitud de proyectos de IoT. Además, ofrece compatibilidad con Arduino, pudiendo utilizar los amplios recursos desarrollados por esa comunidad.
\end{itemize}

\subsubsection*{Sensores}
\begin{itemize}
    \item DHT22:
    \item DSB18B20:
    \item LDR:
    \item YL-69:
\end{itemize}

\subsection*{Lenguaje de programación}

Dado que la microcontroladora elegida ha sido la ESP32, las opciones a la hora de programar disponibles son:

\begin{itemize}
    \item MicroPython: consiste en un pequeño intérprete de Python, el cual contiene un subconjunto mínimo y optimizado de librerias para que pueda correr en microcontroladores.
    En un primer momento, fue la idea elegida para utilizar en nuestro proyecto.
    
    Ventajas:
    \begin{itemize}
        \item RELP Interactiva, es decir, un programa que permite leer e interpretar los comandos para evaluarlos e imprimir el resultado sin la necesidad de compilar ni cargar el programa en el microcontrolador.
        \item Gran cantidad de librerías disponibles
        \item Extensibilidad. Ofrece la posibilidad de mezclas códigod que requieran una ejecución más rápida a bajo nivel mediante la extensión de sus funciones.
    \end{itemize}
    
    Desventajas:
    \begin{itemize}
        \item Código más lento y con necesidad de mayor cantidad de memoria, en comparación con C o C++
        \item Proceso de inicialización del microcontrolador más complejo
        \item Funcionalidad de ciertas librerías para componentes más limitadas
    \end{itemize}
     
    \item RTOS: es un sistema de operación en tiempo realutilizado en sistemas embebidos. Proporciona métodos para múltiples subprocesos o hilos, mutexes, semáforos, temporizadores... Además de soportar las prioridades de los hilos. Por último, cuenta con un modo que reduce el consumo energético (tickless). En FreeRTOS las aplicaciones puede asignarse de manera estática, mientras que los objetos pueden asignarse dinámicamente con distintos esquemas de asignación de memoria. 
    Pese a contar con una enorme cantidad de documentación, la complejidad y la curva de aprendizaje que supondría el uso de este lenguaje hizo que fuera descartado rápidamente.
    
    \item Mongoose OS: framework de desarrollo disponible bajo la licencia de Apache (con una versión community y otra enterprise) y amplaimente utilizado en aplicaciones relacionadas con el IoT. Cuenta con compatibilidad para microcontroladores de bajo consumo, como es el caso del ESP32. Cuenta con un servidor web integrado, soporta programación tanto en C como en JavaScript y cuenta con integración de nubes privadas y públicas (como AWS Iot o Mosquitto).
    
    \item IDE Arduino:  Software de código abierto que cuenta con una amplia comunidad de respaldo. Los lenguajes que admite son C y C++ haciendo uso de reglas especiales de estructuración de código, además de que bajo este IDE se suministran multitud de bibliotecas para procedimientos comunes de entrada y salida. Finalmente ha sido la opción elegida para la programación de la ESP32, ya que la forma de empezar a usar este software con la placa es muy sencilla.
    
\end{itemize}

\section{Metodología de desarrollo}
La metodología escogida para la realización del proyecto ha sido Lean más Kanban con ligeras modificaciones. Los motivos que impulsaron la elección de esta metodología fueron los principios que sigue:

\begin{enumerate}
\item Eliminación de desperdicios. Todo aquello que no aporte valor debe eliminarse (códigos basura, requisitos modificados...)
\item Ampliación del aprendizaje. Aunque inicialmente cada miembro del equipo se encargará del desarrollo de una parte del proyecto, todo el equipo está abierto a ofrecer su ayuda a los demás y a aprender nuevas tecnologías o formas de resolver problemas.
\item Realizar las tomas de decisiones a medida que se van teniendo en cuenta los requisitos a cumplir, así como la manera de enfocarlos. El proyecto ha ido construyéndose de manera progresiva, no se ha seguido una idea fija desde el primer momento sino que esta se ha ido desarrollando y madurando con el tiempo.
\item Entregas rápidas. Cada vez que se realiza una entrega esta implica un aumento de la funcionalidad o una corrección de esta.
\item Potenciar el equipo, todos los miembros han sido partícipes en la toma de decisiones importantes para el proyecto.
\item Creación de integridad, haciendo uso de un sistema para el control de versiones (github en nuestro caso).
\item Visualizar todo en conjunto. \textit{“Pensar en grande, actuar en pequeño, equivocarse rápido y aprender con rapidez”} Esa frase resume la mentalidad de todos los integrantes del equipo.
\end{enumerate}

Esta metodología se ha apoyado con Kanban, el cual no es una metodología de trabajo en sí sino una forma de visualizar el trabajo, controlar la asignación de tareas y mejorar la comunicación de los miembros del equipo.

\subsection{Aplicación de la metodología}
Para aplicar la metodología de manera eficaz se realizaba semanalmente una breve exposición de los avances realizados (mejoras realizadas, problemas encontrados...), así como un nuevo reparto de tareas y la exposición de alguna nueva idea aplicable al proyecto.

Todo la actividad quedaba registrada en github mediante los commit y el dashboard con las distintas actividades.

\section{Arquitectura de la aplicación}
De verdad que te lo digo, Dave is a love.

\section{Modelo de negocio}
A very big loveee loveeee.

\section{Planificación temporal y de desarrollo}
uuuuuh uuuuuh yellow submarine yellow submarine yes, yellow submarine

\section{Riesgos del proyecto}
avia huna bhez un vaarkito chikitito TT

\section{Plan de contingencias}


\section{Resumen y conclusiones}
La Antorcha Humana fue a bankia y le denegaron una hipoteca.

\end{document}


