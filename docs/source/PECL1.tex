\documentclass[runningheads]{llncs}
% Hipervínculos
\usepackage{hyperref}
% Tablas
\usepackage[table,xcdraw]{xcolor}
% Idioma y tildes
\usepackage[spanish]{babel}
\selectlanguage{spanish}
\usepackage[utf8]{inputenc}

% Listas sencillas
\usepackage[ampersand]{easylist}


%%%%%%%%%%%%%%%%%%%%%%%%%%%%%%%%%%%%%%%%%%%%%%%%%%%%%%%%%%%%%%%%%%%%%%%%%%%%%%%

\title{PECL1. Ubiquitous Computation} % The name of the project.
\author{
    Pablo Acereda García \and 
    David Emanuel Craciunescu \and 
    Pablo Martínez Gracia \and 
    Laura Pérez Medeiro}

\date{October 2019}

\begin{document}

\maketitle

%%%%%%%%%%%%%%%%%%%%%%%%%%%%%%%%%%%%%%%%%%%%%%%%%%%%%%%%%%%%%%%%%%%%%%%%%%%%%%%

\section{Introduction}
The project will consist of an intelligent system with capabilities to control
abiotic factors that affect the wellbeing of plants, with the objective of
improving their health and quality of live.

Factors such as humidity and temperature, as well as the levels of light these
receive, will be monitorized and analyzed frequently in order to verify and 
ensure optimal quality of life for the plants.

Thanks to these measurements, a series of alarms and different warning states
will be implemented in order to better control the status of the plants and
alert the users of their current situation.

Users will be able to easily install the system themselves, and interact and
control it via a virtual assistant such as \textit{Amazon Alexa}.
    
%%%%%%%%%%%%%%%%%%%%%%%%%%%%%%%%%%%%%%%%%%%%%%%%%%%%%%%%%%%%%%%%%%%%%%%%%%%%%%%

\section{Context}

    %%%%%%%%%%%%%%%%%%%%%%%%%%%%%%%%%%%%

    \subsection{Current situation of the presented problem}

    The market has seen its fair share of intelligent systems created to aid
    with plant irrigation control, and there are those that even come with a
    mobile app interface, such as Blossom, PlantLink or Edyn. Some of the
    newcomers to jump aboard the backyard-sprinkler train are the so called 
    `intelligent flowerpots', such as the one \textit{Xiaomi} has started to 
    sell recently, capable of watering the plants automatically depending on the
    humidity of the soil.

    There are other projects like \textit{GR0} capable of suggesting what plant
    species one should buy by analyzing the quality and type of soil one has.

    \newline
    
    Nevertheless, none of these offers use any kind of virtual-assistant
    integration or any well-designed user experience, for that
    matter. \textit{That} will be the main difference our project will have. Not
    only will the user control the system through a virtual assistant, 
    the design and user experience is planned to be exceptional and extremely 
    intuitive.

    %%%%%%%%%%%%%%%%%%%%%%%%%%%%%%%%%%%%
   
    \subsection{End-of-project projection}
   
    Once the course finishes, our intention is to have created a device capable
    of automatically keeping alive crops or plants with interactions through
    \textit{Amazon Alexa}.

    In order to achieve that, our device will be able to give accurate weather
    predictions, alert of possible plagues that might attack the crops and
    monitor the optimal conditions for the plants themselves.
    
    %%%%%%%%%%%%%%%%%%%%%%%%%%%%%%%%%%%%

    \subsection{Targed audience}
    
    This project aims to aid the gardening aficionados that do not have a great
    amount of time at their disposal to take care of their plants optimally. It
    also aims to assist farmers that need help with the care of their crops and
    seek to minimize the effect of unexpected and external factors to their
    crops.

\section{Alcance del proyecto}

\section{Descripción de ideas descartadas}

\section{Tecnología a utilizar}
Las decisiones acerca de la tecnología a utilizar en el proyecto se han tomado 
teniendo en cuenta la experiencia de los distintos miembros que conforman el 
área de desarrollo, así como el costo que tienen los distintos componentes.

\newline
Las distintas opciones barajadas para la realización del proyecto han sido:

\subsection*{Hardware}
\subsubsection*{Controladoras}
\begin{itemize}
    \item Arduino: Ofrece una amplia gama de placas, con distintas prestaciones, cuenta con un entorno de programación para la creación de aplicaciones y una gran comunidad que apoya el Desarrollo. El principal inconveniente que se encontraba era su precio, el cual ronda los 20 euros.
    
    \item Raspberry: Fue una opción que se estuvo barajando utilizar, ya que ofrece una amplia variedad de lenguajes de programación a utilizar. Sin embargo, fue descartada debido a su gran tamaño.
    
    \item ESP32: Ha sido la opción finalmente elegida, ya que se trata de un SoC de muy bajo coste (su precio ronda los 5 euros), cuenta con Bluetooth y es utilizado en multitud de proyectos de IoT. Además, ofrece compatibilidad con Arduino, pudiendo utilizar los amplios recursos desarrollados por esa comunidad.
\end{itemize}

\subsubsection*{Sensores}
\begin{itemize}
    \item DHT22:
    \item DSB18B20:
    \item LDR:
    \item YL-69:
\end{itemize}

\subsection*{Lenguaje de programación}

Dado que la microcontroladora elegida ha sido la ESP32, las opciones a la hora de programar disponibles son:

\begin{itemize}
    \item MicroPython: consiste en un pequeño intérprete de Python, el cual contiene un subconjunto mínimo y optimizado de librerias para que pueda correr en microcontroladores.
    En un primer momento, fue la idea elegida para utilizar en nuestro proyecto.
    
    Ventajas:
    \begin{itemize}
        \item RELP Interactiva, es decir, un programa que permite leer e interpretar los comandos para evaluarlos e imprimir el resultado sin la necesidad de compilar ni cargar el programa en el microcontrolador.
        \item Gran cantidad de librerías disponibles
        \item Extensibilidad. Ofrece la posibilidad de mezclas códigod que requieran una ejecución más rápida a bajo nivel mediante la extensión de sus funciones.
    \end{itemize}
    
    Desventajas:
    \begin{itemize}
        \item Código más lento y con necesidad de mayor cantidad de memoria, en comparación con C o C++
        \item Proceso de inicialización del microcontrolador más complejo
        \item Funcionalidad de ciertas librerías para componentes más limitadas
    \end{itemize}
     
    \item RTOS: es un sistema de operación en tiempo realutilizado en sistemas embebidos. Proporciona métodos para múltiples subprocesos o hilos, mutexes, semáforos, temporizadores... Además de soportar las prioridades de los hilos. Por último, cuenta con un modo que reduce el consumo energético (tickless). En FreeRTOS las aplicaciones puede asignarse de manera estática, mientras que los objetos pueden asignarse dinámicamente con distintos esquemas de asignación de memoria. 
    Pese a contar con una enorme cantidad de documentación, la complejidad y la curva de aprendizaje que supondría el uso de este lenguaje hizo que fuera descartado rápidamente.
    
    \item Mongoose OS: framework de desarrollo disponible bajo la licencia de Apache (con una versión community y otra enterprise) y amplaimente utilizado en aplicaciones relacionadas con el IoT. Cuenta con compatibilidad para microcontroladores de bajo consumo, como es el caso del ESP32. Cuenta con un servidor web integrado, soporta programación tanto en C como en JavaScript y cuenta con integración de nubes privadas y públicas (como AWS Iot o Mosquitto).
    
    \item IDE Arduino:
    
    
\end{itemize}

\section{Metodología de desarrollo}

\section{Arquitectura de la aplicación}

\section{Modelo de negocio}

\section{Planificación temporal y de desarrollo}

\section{Riesgos del proyecto}

\section{Plan de contingencias}

\section{Resumen y conclusiones}

\end{document}
