%%%%%%%%%%%%%%%%%%%%%%%%%%%%%%%%%%%%%%%%%%%%%%%%%%%%%%%%%%%%%%%%%%%%%%%%%%%%%%%%
% Ubiquitous Computation - Lab Practice 1                                      %
%   - Pablo Acereda García                                                     %
%   - David Emanuel Craciunescu                                                %
%   - Pablo Martínez Gracia                                                    %
%   - Laura Pérez Medeiro                                                      %
%%%%%%%%%%%%%%%%%%%%%%%%%%%%%%%%%%%%%%%%%%%%%%%%%%%%%%%%%%%%%%%%%%%%%%%%%%%%%%%%

%%%%%%%%%%%%%%%%%%%%%%%%%%%%%%%%%%%%%%%%%%%%%%%%%%%%%%%%%%%%%%%%%%%%%%%%%%%%%%%%
%                                                                              %
%                               LaTeX Formatting                               %
%                                                                              %
%%%%%%%%%%%%%%%%%%%%%%%%%%%%%%%%%%%%%%%%%%%%%%%%%%%%%%%%%%%%%%%%%%%%%%%%%%%%%%%%

% !TeX spellcheck   = en-US
% !TeX encoding     = utf8
% !TeX program      = pdflatex
% !BIB program      = bibtex
% -*- coding:utf-8 mod:LaTeX -*-

% "a4paper" enables:
%  - easy print out on DIN A4 paper size

% English documents: pass english as class option
\documentclass[english,runningheads,a4paper]{llncs}[2018/03/10]
\usepackage[ngerman,main=english]{babel}
\addto\extrasenglish{\languageshorthands{ngerman}\useshorthands{"}}

\usepackage{regexpatch}
\makeatletter
\edef\switcht@albion{%
  \relax\unexpanded\expandafter{\switcht@albion}%
}
\xpatchcmd*{\switcht@albion}{ \def}{\def}{}{}
\xpatchcmd{\switcht@albion}{\relax}{}{}{}
\edef\switcht@deutsch{%
  \relax\unexpanded\expandafter{\switcht@deutsch}%
}
\xpatchcmd*{\switcht@deutsch}{ \def}{\def}{}{}
\xpatchcmd{\switcht@deutsch}{\relax}{}{}{}
\edef\switcht@francais{%
  \relax\unexpanded\expandafter{\switcht@francais}%
}
\xpatchcmd*{\switcht@francais}{ \def}{\def}{}{}
\xpatchcmd{\switcht@francais}{\relax}{}{}{}
\makeatother

\usepackage{ifluatex}
\ifluatex
  \usepackage{fontspec}
  \usepackage[english]{selnolig}
\fi

\iftrue % use default-font
  \ifluatex
    \setmainfont{Latin Modern Roman}
    \setsansfont{Latin Modern Sans}
    \setmonofont{Latin Modern Mono} % "variable=false"
  \else
    \usepackage[%
      rm={oldstyle=false,proportional=true},%
      sf={oldstyle=false,proportional=true},%
      tt={oldstyle=false,proportional=true,variable=false},%
      qt=false%
    ]{cfr-lm}
  \fi
\else
  \ifluatex
    \setmainfont{TeX Gyre Termes}
    \setsansfont[Scale=.9]{TeX Gyre Heros}
    \setmonofont{Latin Modern Mono} % "variable=false"
  \else
    \usepackage{newtxtext}
    \usepackage{newtxmath}
    \usepackage[zerostyle=b,scaled=.9]{newtxtt}
  \fi
\fi

\ifluatex
\else
  \usepackage[T1]{fontenc}
  \usepackage[utf8]{inputenc} %support umlauts in the input
\fi

\usepackage{graphicx}
\usepackage{upquote}
\usepackage{booktabs}
\usepackage{paralist}
\usepackage{csquotes}
\usepackage{textcmds}

% Enable margin kerning
\RequirePackage[%
  babel,%
  final,%
  expansion=alltext,%
  protrusion=alltext-nott]{microtype}%
\DisableLigatures{encoding = T1, family = tt* }

\usepackage{url}
\makeatletter
\g@addto@macro{\UrlBreaks}{\UrlOrds}
\makeatother

% Required for package pdfcomment later
\usepackage{xcolor}

% For listings
\usepackage{listings}
\lstset{%
  basicstyle=\ttfamily,%
  columns=fixed,%
  basewidth=.5em,%
  xleftmargin=0.5cm,%
  captionpos=b}%
\renewcommand{\lstlistingname}{List.}
\usepackage{chngcntr}
\AtBeginDocument{\counterwithout{lstlisting}{section}}

% Enable nice comments
\usepackage{pdfcomment}

\newcommand{\commentontext}[2]{\colorbox{yellow!60}{#1}\pdfcomment[color={0.234 0.867 0.211},hoffset=-6pt,voffset=10pt,opacity=0.5]{#2}}
\newcommand{\commentatside}[1]{\pdfcomment[color={0.045 0.278 0.643},icon=Note]{#1}}

\newcommand{\todo}[1]{\commentatside{#1}}
% Compatiblity with package fixmetodonotes
\newcommand{\TODO}[1]{\commentatside{#1}}

% Bibliography
\ifluatex
\else
  \SetExpansion
  [ context = sloppy,
    stretch = 30,
    shrink = 60,
    step = 5 ]
  { encoding = {OT1,T1,TS1} }
  { }
\fi

% Put footnotes below floats
% Source: https://tex.stackexchange.com/a/32993/9075
\usepackage{stfloats}
\fnbelowfloat

\usepackage{hyperref}
\hypersetup{hidelinks,
  colorlinks=true,
  allcolors=black,
  pdfstartview=Fit,
  breaklinks=true}

% Enable correct jumping to figures when referencing
\usepackage[all]{hypcap}

\usepackage[group-four-digits,per-mode=fraction]{siunitx}

\usepackage[capitalise,nameinlink]{cleveref}

% Nice formats for \cref
\usepackage{iflang}
\IfLanguageName{ngerman}{
  \crefname{table}{Tab.}{Tab.}
  \Crefname{table}{Tabelle}{Tabellen}
  \crefname{figure}{\figurename}{\figurename}
  \Crefname{figure}{Abbildungen}{Abbildungen}
  \crefname{equation}{Gleichung}{Gleichungen}
  \Crefname{equation}{Gleichung}{Gleichungen}
  \crefname{listing}{\lstlistingname}{\lstlistingname}
  \Crefname{listing}{Listing}{Listings}
  \crefname{section}{Abschnitt}{Abschnitte}
  \Crefname{section}{Abschnitt}{Abschnitte}
  \crefname{paragraph}{Abschnitt}{Abschnitte}
  \Crefname{paragraph}{Abschnitt}{Abschnitte}
  \crefname{subparagraph}{Abschnitt}{Abschnitte}
  \Crefname{subparagraph}{Abschnitt}{Abschnitte}
}{
  \crefname{section}{Sect.}{Sect.}
  \Crefname{section}{Section}{Sections}
  \crefname{listing}{\lstlistingname}{\lstlistingname}
  \Crefname{listing}{Listing}{Listings}
}

% Solution for hyperlink refs.
\newcommand{\Vlabel}[1]{\label[line]{#1}\hypertarget{#1}{}}
\newcommand{\lref}[1]{\hyperlink{#1}{\FancyVerbLineautorefname~\ref*{#1}}}
    
\usepackage{xspace}
%\newcommand{\eg}{e.\,g.\xspace}
%\newcommand{\ie}{i.\,e.\xspace}
\newcommand{\eg}{e.\,g.,\ }
\newcommand{\ie}{i.\,e.,\ }

% Powerset
\DeclareFontFamily{U}{MnSymbolC}{}
\DeclareSymbolFont{MnSyC}{U}{MnSymbolC}{m}{n}
\DeclareFontShape{U}{MnSymbolC}{m}{n}{
  <-6>    MnSymbolC5
  <6-7>   MnSymbolC6
  <7-8>   MnSymbolC7
  <8-9>   MnSymbolC8
  <9-10>  MnSymbolC9
  <10-12> MnSymbolC10
  <12->   MnSymbolC12%
}{}
\DeclareMathSymbol{\powerset}{\mathord}{MnSyC}{180}

% Name says it all.
\ifluatex
\else
  \input glyphtounicode
  \pdfgentounicode=1
\fi

% Correct bad hypenation.
\hyphenation{op-tical net-works semi-conduc-tor}

% Some useful info to add.

\iffalse
  \usepackage[intended]{llncsconf}
  \conference{name of the conference}
  \llncs{book editors and title}{0042} %% 0042 is the start page
\fi

% For demonstration purposes only
\usepackage[math]{blindtext}
\usepackage{mwe}
\usepackage[backend=biber, style=numeric]{biblatex}
\addbibresource{java.bib}
\usepackage[ampersand]{easylist}

%%%%%%%%%%%%%%%%%%%%%%%%%%%%%%%%%%%%%%%%%%%%%%%%%%%%%%%%%%%%%%%%%%%%%%%%%%%%%%%%
%                                                                              %
%                               Actual content                                 %
%                                                                              %
%%%%%%%%%%%%%%%%%%%%%%%%%%%%%%%%%%%%%%%%%%%%%%%%%%%%%%%%%%%%%%%%%%%%%%%%%%%%%%%%

\title{ec\textbf{\o}, a Rainforest Product}
\author{
    Pablo Acereda García \and
    David Emanuel Craciunescu \and
    Pablo Martínez Gracia \and
    Laura Pérez Medeiro
}
\date{October 2019}

\begin{document}

\maketitle

%%%%%%%%%%%%%%%%%%%%%%%%%%%%%%%%%%%%%%%%%%%%%%%%%%%%%%%%%%%%%%%%%%%%%%%%%%%%%%%%

\section*{Introduction}

This project will consist of an intelligent system with capabilities to control
abiotic factors that affect the wellbeing of plants, with the objective of
improviing their health and quality of life.

Factors such as humidity and temperature, as well as the levels of light these
receive, will be monitorized and analyzed frequently in order to verify and
ensure optimal quality of life for the plants.

Thanks to these measurements, a series of alarms and different warning states
will be implemented in order to better control the status of the plants and
alert the users of their current situation.

Users will be able to easily install the system themselves, and interact and
control it via a virtual assistant such as \textit{Amazon Alexa}.

%%%%%%%%%%%%%%%%%%%%%%%%%%%%%%%%%%%%%%%%%%%%%%%%%%%%%%%%%%%%%%%%%%%%%%%%%%%%%%%%

\section*{Context}

    %%%%%%%%%%%%%%%%%%%%%%%%%%%%%%%%%%%%

    \subsection*{Current situation of the presented problem}

    The market has seen its fair share of intelligent systems created to aid
    with plant irrigation control, and there are those that even come with a
    mobile app interface, such as Blossom, PlantLink or Edyn. Some of the
    newcomers to jump aboard the backyard-sprinkler train are the so-called
    `intelligent flowerpots', like the one \textit{Xiaomi} has recently started
    to sell, capable of watering the plants automatically depending on the
    humidity of the soil.

    There are other projects like \textit{GR0} that are capable of suggesting
    what plant species one should buy by analyzing the quality and type of soil
    one uses.

    Nevertheless, none of these offers use any kind of virtual-assistant
    integration or any well-designed user experience, for that matter.
    \textit{That} will be the main difference our project will have. Not only
    will the user control the system through a virtual assistant, the design and
    user experience is planned to be exceptional and extremely easy and
    intuitive.

    %%%%%%%%%%%%%%%%%%%%%%%%%%%%%%%%%%%%

    \subsection*{End-of-project prediction}

    Once the course finishes, our intention is to have created a device capable
    of auomatically keeping alive crops or plants with interactions through
    \textit{Amazon Alexa}.

    In order to achieve that, our device will be able to give accurate weather
    predictions, alert of possible plagues that might attack the crops and
    monitor the optimal conditions for the plants themselves.

    %%%%%%%%%%%%%%%%%%%%%%%%%%%%%%%%%%%%

    \subsection*{Target audience}

    This project aims to aid the gardening aficionados that do not have a great
    amount of time at their disposal to take care of their plants optimally. It
    also aims to assist farmers that need help with the care of their crops and
    seek to minimize the effect of unexpected and external factors to their
    produce.

%%%%%%%%%%%%%%%%%%%%%%%%%%%%%%%%%%%%%%%%%%%%%%%%%%%%%%%%%%%%%%%%%%%%%%%%%%%%%%%%

\section*{Project Scope}

Just as previously mentioned, the main objective of this technological endeavor
is the creation of a good-enough prototype in order to shot to multiple possible
future investors so that they help to commercialize and empower the creation of
the product itself through their monetary support. In short and plain English,
the objective of the product is to attract suitable investors that, with their
contributions, would help the product grow and transform into a marketable
commercial system. This product, `\textit{ec\textbf{\o}}', is currently the
only product the startup \textit{Raninforest} has, therefore this project is of
vital importance to the company.

`\textit{ec\textbf{\o}}' is a product than can be commercialized both as a
physical system and a digital solution for the client. As a physical system it
will be able to monitor all the ambiental factors that could affect a plant's
growth and development, e.g.\ humidity, temperature, luminosity, soil quality
etc.

As a digital solution, it comes with visualization software that will be able to
show and display information in a clear and concise way, as well as in an
intuitive manner. The project promoters also hope to be able to motivate a big
community of collaborators so that the product improves substantially over time.

This is not all though, for the `\textit{ec\textbf{\o}}' also comes with Virtual
Assistant integration, with which the user would be able to interact easily and
seamlessly.

When it comes to future functions or ones in early stages of development, the
biggest one so far has been the incorporation of sensors and mechanisms capable
of detecting the pressence of bugs and insects that could pose a threat to the
plant. There are multiple other initiatives and lines of development and
improvement for the product within the company, but those are in very early
stages and developing further into those lines of work in this document would
not be appropriate, given the volatile nature of initiatives in such early
stages.

%%%%%%%%%%%%%%%%%%%%%%%%%%%%%%%%%%%%%%%%%%%%%%%%%%%%%%%%%%%%%%%%%%%%%%%%%%%%%%%%

\section*{Discarded Ideas}

The project has suffered many changes since its intial conception and ideation.
Some of these changes and modifications have been so great and have influenced
the project such a big deal that they earn a mention in the following list, just
by their very nature as disruptive and polarizing within the developing team:

\begin{easylist}[itemize]

& \textbf{Detection of plagues} \\
An initial idea was that the system should detect different insects and monitor
for known plagues that attack plants and crops. Conceptually insect detection
seems simple but its implementation isn't. Some initial ideas entailed the used
of infrared radiation, but this idea turned out to be completely inefficient.

This possible feature has not been completely discarded just yet. There is a
type of sensor called \textit{PIR}, which could be configured in such a way that
detect small-sized animals. The development of this idea is currently
experimenting some issues.

& \textbf{Development of own microcontroller} \\
This would have been extremely useful for the product itself, given that the
development of a microcontroller specialized for the task would have optimized
the way in which the product was designed. After much thought, this idea was
discarded, given the short timeframe this product has for development and the
ambitious nature of the initiative.

\end{easylist}

%%%%%%%%%%%%%%%%%%%%%%%%%%%%%%%%%%%%%%%%%%%%%%%%%%%%%%%%%%%%%%%%%%%%%%%%%%%%%%%%

\section*{Technology to Use}

This section contains the different options that were considered for the
project. The different decisions about the different used technology within ght
project took into account the experience of the various members of the group, as
well as the monetary cost of the different hardware elements, their general
availability and the trust the very manufacturer inspired.

%%%%%%%%%%%%%%%%%%%%%%%%%%%%%%%%%%%%%%%%

    \subsection*{Controllers}

    \begin{easylist}[itemize]

    & \textbf{Arduino}

    \textit{Arduino} is an open-source electronics platform that is based on
    easy-to-use software and hardware. The \textit{Arduino} boards themselves 
    are controlled by sending a set of instructions to a microcontroller on the 
    board. To do so, one must use the \textit{Arduino} programming language, 
    which is based on \textit{Wiring}, and the \textbf{Arduino Software (IDE)}, 
    which is based on \textit{Processing}.

    These boards were our first choice because they are extremely easy to use. 
    The programming language is extremely easy to pick up if one already knows
    \textit{C}, and the \textit{Arduino} community offers a wide array of
    free-to-use resources that improve the quality and reduce the effort of any 
    project trying to use \textit{Arduino}.

    Even with all it's benefits, \textit{Arduino} ended up being discarded as a
    possible option, given that the absolutely cheapest of boards would still 
    cost around \$20.

    & \textbf{Raspberry Pi}

    According to the official \textit{Raspberry Pi} webpage: \textit{``Raspberry
    Pi is a low cost, credit-card sized computer that plugs into a monitor or a 
    TV, and uses standard keyboard and mouse. (\ldots) It's capable of doing 
    what you'd normally expect a regular computer to do, from browsing the 
    internet and playing high-definition video, to making spreadsheets, word 
    processing and playing games''}.

    Just like a regular computer, once would be able to program it to do 
    whatever they'd want it to do. This would have been ideal for the project 
    itself, given the simplicity of its use and the gigantic array of languages 
    that are compatible with it. In the end, though, it was discarded because 
    the computer itself was too large in comparison to the rest of the elements 
    of the project.

    & \textbf{ESP32}

    The \textit{ESP32} is a series of low-cost, low-power system on a chip
    \textbf{SoC} microcontrollers with integrated Wi-Fi and dual-mode Bluetooth.
    This was the option the team ended up choosing for the project, given that 
    the \textit{ESP32} was specifically designed for \textbf{wearable 
    electronics and IoT applications.}

    One can also program on it using the \textit{Arduino IDE}, which was a huge
    advantage. Most of the team already knew how to use and control
    \textit{Arduino}, so not having to learn a skil specifically for the
    microcontroller plus all the benefits of the \textit{Arduino} community made
    the group decide on the \textit{ESP32} as an option.

    \end{easylist}

%%%%%%%%%%%%%%%%%%%%%%%%%%%%%%%%%%%%%%%%

    \subsection*{Sensors}

    There are many different kinds of sensors on the market. Many of these are
    extremely capable and can be obtained very easily. When deciding on the 
    sensors, the team always looked for the most reliable and available on the 
    market. Here are some of the sensors the project will use:

    \begin{easylist}[itemize]

    & \textbf{DHT22} \\
    It is a basic and low-cost digital temperature and humidity sensor. It 
    measures the surrounding air thanks to a capacity humiditor sensor and a
    thermistor, and spits out digital signal on the data pin. It's extremely 
    simple to use, given that its transmissions are already digital, but one can
    only make readings each two seconds. Extreme speed, or lack thereof, is not
    a highly important factor when it comes to plant humidity and temperature
    measuring, so the only real downside would not be an actual downside. 

    Just like any avid and shrewd reader would have realized by now, this sensor
    will be used to measure exterior temperature and humidity.

    & \textbf{LDR} \\
    A \textit{Light Dependent Resistor} or a photo resistor is a device whose
    resistivity is a function of the incident electromagnetic radiation. These
    are light sensitive devices also called photo conductors, photo conductive
    cells or simply photocells.

    The ability to detect light is a must in the soon-to-be deliverable product,
    therefore the use of this sensor was an easy and obvious solution.

    & \textbf{YL-69} \\
    According to the product description the manufacturer provides:
    \textit{``The YL-69 is a simple sensor that can be used to detect soil
    moisture/relative humidity within the soil. The module is able to detect
    when the soil is too dry or wet. Great for use with automatic plant watering
    systems''}.

    The need for a sensor like this, especially considering it's insignificant
    price in comparison to the price of the rest of the setup and components.
    Terefore, this will be the main triggering mechanism to notify the plant or
    plants need to be watered.

    \end{easylist}

%%%%%%%%%%%%%%%%%%%%%%%%%%%%%%%%%%%%%%%%

    \subsection*{Programming Languages}

    As previously mentioned, the chosen microcontroller is the \textit{ESP32}.
    Therefore, the options when it comes to programming languages have been:

    \begin{easylist}[itemize]
    
    & MicroPython

    \textit{MicroPython} an open source \textit{Python} programming language
    interpreter that is capable of running on small embedded development boards.
    With the adaptability build into \textit{MicroPython}, one can write clean
    and simple \textit{Python} code to control hardware directly instead of
    having to use complex low-level languages.

    Even if a reduced version, \textit{MicroPython} still supports most of
    \textit{Python}'s syntax and implements most of its inner mechanism. Given
    all these features and advantages, it was a quick initial choice for the
    project.

    These are some of the features that set it apart from other embedded
    systems:

        && \textbf{Interactive REPL}
    This feature allows execution of code without the need of any compilation or
    uploading time, which is perfect for systems with a high level of
    experimentation.

        && \textbf{Extensive software library}
    \textit{MicroPython} already comes with libraries built in to support common
    tasks, like JSON data parsing, regular expression handling or even network
    socket programming.

        && \textbf{Extensibility}
    Advanced users ma be able to mix \textit{MicroPython} with extensible
    low-level C/C++ functions in order to further optimize their code and make
    the execution faster when it really matters.

    \textit{MicroPython} has some very useful features. Unfortunately, it also
    comes with its downsides:

        && Slower code and higher memory needs when compared to C/C++.
        && Complicated microcontroller initialization process.
        && Limited functionality for some key libraries.

    & FreeRTOS

    \textit{FreeRTOS} is a real-time operating system made specifically to run 
    on embedded systems. It is especially good because it natively provides core
    real time scheduling, inter-task communication, timing and syncrhonisation
    primitives. This transalates into a more accurately controlled kernel and a
    system that is able to execute tasks exactly when they have to be executed, 
    a \textit{deterministic} system.

    In \textit{FreeRTOS}, applications can be assigned in a static manner while
    objects themselves can be assigned dynamically with different 
    memory-assignment memory schemes. This system was quickly discarded. Even 
    with the great deal of documentation it counts, the complexity of 
    \textit{FreeRTOS} and the learning curve it would entail made it 
    practiacally impossible in the provided timeframe.

    & Mongoose OS

    \textit{Mongoose OS} is an Internet of Things Firmware Development Framework
    under Apache License Version 2.0.

    \textit{Mongoose OS} is a firmware development framework specifically 
    designed for development of IoT products. It is highly compatible with a 
    wide array of microcontrollers, just like the \textit{ESP32}, and it's 
    objective is to fill `the noticeable for embedded software developers' 
    between firmware created for prototyping and bare-metal microcontrollers.

    It comes with an integrated web server and supports interaction with both
    private and public clouds like AWS IoT or Mosquitto.

    & Arduino IDE

    It's an open-source project that has a very large community behind it.
    \textit{Arduino IDE} comes with a text editor for writing code, a message 
    area, a console, and other common functionalities of full-fledged IDEs. The
    programming languages it admits are C and C++, but with some special rules 
    when it comes to structuring the code. The IDE also comes with plenty of 
    built-in libraries for common tasks like robot control or I/O mechanisms.

    This is the option the team ended up choosing, not only for its simplicity 
    and usefulness, but also because it can interact directly with the 
    \textit{ESP32} chip that the project will use as a main component.

    \end{easylist}

%%%%%%%%%%%%%%%%%%%%%%%%%%%%%%%%%%%%%%%%%%%%%%%%%%%%%%%%%%%%%%%%%%%%%%%%%%%%%%%%

\section*{Development Methodology}

One of the key factors when it comes to a project, especially the creation of a
new one, is the \textit{Development Methodology}. These methodologies are a
means by which a product can be created in an organized, somewhat predictable
and planned manner. A \textit{Development Methodology} is nothing more than the
process of planning, creating, testing and then deploying a project. And the
different types of methodologies only alter the way in which those components
interact with each other. 

The specific methodology used for this project is \textit{Lean w/ Kanban}. Here
are some of the main reasons that made the group choose this methodology:

\begin{easylist}[enumerate]

& \textbf{Waste reduction}. Helps keep project simple and clean. Everything that
is not absolutely necessary is promptly eliminated, be it garbage code,
unecessary meetings, obsolete hardware, etc.

& \textbf{Knowledge improvement}. This is one of the best aspects of
\textit{Lean}. Even if every member has a defined function within the team and
their tasks are clearly stated, any other member can lend a hand and help if
they feel the need or have the time. This helps keep the team agile and on their
feet.

& \textbf{Delayed decision-making}. Within \textit{Lean} decisions are made as
the requirements are being met, as well as their focus starts to change. The
project starts building itself in a progressive and adaptative manner because it
hasn't followed a strictly defined idea since its inception. Instead, it has
been allowed to grow and morph into it needed to become.

& \textbf{Quick delivery}. Given the minimalism of the methodology and its short
iterations of work, the authors of this document believe a further explanation
of this specific aspect would be an insult to the reader.

& \textbf{Team-oriented}. Every single member is part of the decision-making
process, as well as the definition of their own work. This methodology works
best in teams that extremely motivated because the different members usually
divide the work amongst them without the need of actual meetings and specific,
written-down planification.

& \textbf{Focus on the objective}. Thanks to many of the factors mentionded
above, the project does not lose sight of the objective. One could say the
\textit{motto} of the team would be \textit{``Think big, act small, err
frequently and fix quickly''}.

\end{easylist}

When combining \textit{Lean} with \textit{Kanban}\footnote{Kanban is a
visualization methodology and mechanism, not an actual defined development
process.} controlling the assignment of different tasks and jobs of the
different team members was extremely easy and intuitive. This combination is the
key component to the development of the prototype of the product, since it
increases visibility of the project and enhances control of the direction in
which the development steers towards.

\textit{Lean Kanban} is a methodology used by various startups, just like IMVU,
which is a company created by Eric Ries. He created a methodology called
\textit{Lean Startup}, which as `improved' version of the methodology this
project uses, designed specifically to adapt to the way a startup works.

The development cycles in \textit{Lean Kanban} start with the creation of an
idea, which is then assessed and developed in order to analyze if its addition
to the overall project would be of any real value. With this process, the worse
case scenario is still obtaining knowledge about the creation and implementation
of that specific idea, even if it does not end up being part of the project.
This is the way some of the most important parts of `\textit{ec\textbf{\o}}'
were created, like the \textit{Amazon Alexa} virtual-assistant integration or
the detection of factors like humidity and light levels.

    %%%%%%%%%%%%%%%%%%%%%%%%%%%%%%%%%%%%

    \subsection*{Implementation of the metodology}

    In order to apply the methodology in an effective manner, the team has used
    the project collaboration tools of \textit{GitHub Projects}, which
    officially integrated as part of the software of University of Alcalá thanks
    to the \textit{GitHub Campus Program}.

    The team met weekly to plan their activities and tasks for the following
    week and kept in touch with the mentioned tools, also able to track changes
    in the status of each task and maintain the visibility of the overall
    project.

    Every single modification and all the planification has been recorded thanks
    to \textit{GitHub Projects} and a link to the respository will be provided
    with this document.

%%%%%%%%%%%%%%%%%%%%%%%%%%%%%%%%%%%%%%%%%%%%%%%%%%%%%%%%%%%%%%%%%%%%%%%%%%%%%%%%

\section*{Application Architecture}
%To do: Dear Dave, would you mind to check this part? Please and thank you
The device is formed by two components:
\begin{itemize}
\item Data collector. Formed by all the different sensors that can be found. This component aims to collect and send data to the server. It is the server where the information is going to be treated.
\item Actuator. Formed by the irrigation system, it can be activated when the land's humidity is low or using a voice command. 
\end{itemize}

All the data collected by the sensors is sent to a server, using WiFi and
MQTT protocol. The reasons why this protocol has been chosen are, on the 
one hand, the fact that it is once of the most used protocol in which IoT is referred,
and also the fact that we can use OpenMQTTGateway which provided 
solutions to managed different protocols used by different devices (which means that if there is any sensor that uses BLE, it can be translated into MQTT). 
There is where C takes part, as it is the language in which main.cpp will be created and where all the configuration of MQTT can be found. In the beginning,
data is stored into a relational Database which used PostgreSQL as DBMS. Then, data is sent to the AWS DataBase which will be used by Alexa's API to give
the user information about how the pant is.

Also, the provided information by the sensors will be used in a web page, where register users can find some graphics and statistics.

%%%%%%%%%%%%%%%%%%%%%%%%%%%%%%%%%%%%%%%%%%%%%%%%%%%%%%%%%%%%%%%%%%%%%%%%%%%%%%%%

\section*{Business Model}

When it comes to selling and pitching the product to customers, there are many
factors to take into account. Different elements such as the target demographic
or the country in which the product will be commercialized have to be taken into
account. In order to be successful, the writers of this document have devised a
careful and detailed careful plan in order to optimize the labors of the
company.

There are many things a company must ask itself before selling their products.
The first factor of many is about what the company offers, what value do the
things that the company does have.

%%%%%%%%%%%%%%%%%%%%%%%%%%%%%%%%%%%%%%%%

    \subsection*{Offering. Value Proposition}

    Everyone would love to have plants at home without having to worry about how
    much water they need or whether they are receiving enough sunlight or not.
    Sometimes people buy plants simply to decorate their homes, but have no idea
    how to take care of them, or even consider it before buying for that matter.
    Others may claim to love plants but then have no time, ability or real
    interest whatsoever when it comes to their wellbeing and maintenance.

    `\textit{ec{\textbf{\o}}}' can assure their customers that their plants will
    be well attended at all times. It measures the state of the plant constantly
    by checking its temperature, humidity, pH levels, etc. After collecting all
    this information, it carries out a diagnosis, letting its owners know
    whether the plant is in optimal conditions, whether it's sick or
    not\footnote{Potential feature in development.}, or even if it has any
    nutrient deficiencies in its soil\footnote{Potential feature in
    development.}.

    At the same time, it gives recommendations about the way in which the plant
    would be better off. In the case of lack of water, `\textit{ec\textbf{\o}}'
    will be able to directly water the plants and, if the plant has other needs,
    let the owner know inmediately.

    Furthemore, `\textit{ec\textbf{\o}}' has its own skill on \textit{Amazon
    Alexa}, which makes checking the status of the plants even more convenient.
    Just by asking \textit{Alexa} about the state of the plant the virtual
    assistant will give back to the user all the information related to the
    plant's current state and the needs it has. If the owner does not have
    \textit{Alexa} at home, they can also log in to the \textit{ec\textbf{\o}}
    website in order to check all the information related to their plants.

    To sum up, the product iself adapts to the needs of each plant, providing
    its customers with a simple, time-saving, innovative and customised
    solution. Forgetful or inexperienced customers will not have to worry about
    possibly killing their plants anymore and plant lovers will have detailed
    information about the current state of their plants at all times. Combined
    with the functionalities of \textit{Alexa}, \textit{ec\textbf{\o}} offers an
    even more convenient and efficient way of ensuring the plant's well-being
    while taking care of the planet as a whole.

%%%%%%%%%%%%%%%%%%%%%%%%%%%%%%%%%%%%%%%%

    \subsection*{Customers}

    This section will contain information related to the customers themselves,
    the point of view the product should have when presented to them, different
    `customer segments', etc.

        %%%%%%%%%%%%%%%%%%%%%%%%%%%%%%%%%%%%

        \subsubsection*{Customer segments}

        Nowadays, we all take care of elders, children, animals\ldots But what
        about plants? They are a key factor in the future of the planet and the
        lives of human beings, so there is a general incentive among the
        population to take care of them. By buying \textit{ec\textbf{\o}} this
        task will be easier and quicker than ever, especially nowadays with
        faster work and pace of life. However, it is well-known that not
        everyone is going to be interested in \textit{ec\textbf{\o}}. That's the
        reason why a segmentation strategy must be carried out in order to find
        the product's targets customers.

        By following a segmentation strategy focused on lifestyle, owner's
        income and individual/business approach one can easily perceive that
        \textit{ec\textbf{\o}} will stick to a niche unnatended market formed by
        people that love plants and have them at home as part of their
        lifestyles, businesses whose main activity is related to plants or
        enterprises in sectors such as horticulture, agriculture or gardening.
        Another option would be targeting public spaces and entities such as
        governments, public parks, botanic gardens or governmental buildings.

        %%%%%%%%%%%%%%%%%%%%%%%%%%%%%%%%%%%%

        \subsubsection*{Channels}

        A company can deliver its value proposition to its targeted customers
        through different channels. Reaching each different segment with
        different communication strategies is not only important, but vital to
        the process. Focusing on business-to-business (B2B) and deals with
        governments is recommended since they normally go hand-in-hand with
        discounts due to large volume purchases and business negotiations are at
        stake.

        When it comes to sales to clients for individual use, a large range of
        options are available. In order to attract the attention of those who
        would be willing to buy \textit{ec\textbf{\o}}, the company could
        promote itself through advertisements on specialized magazines such as
        `Country Gardens', `Fine Gardening', or `Horticulture', which are read
        by people interested in plants and probably in \textit{ec\textbf{\o}} as
        well. Another option would be to create online advertisements through
        gardening forums and blogs, social media such as Facebook groups formed
        by people interested in the plant worls, horticulture Instagram
        accounts, etc.

        Furthermore, flyers explaining \textit{ec\textbf{\o}}'s benefits and
        characteristics could be distributed in botanic gardens or
        plant-specialized stores. Indirect advertisements played on screens
        during public events related to gardening could be very efficient to
        find \textit{ec\textbf{\o}}'s target customers as well as word-of-mouth
        among friends, relatives and colleagues who have bought the product and
        are satisfied with it.

        \textit{ec\textbf{\o}} will only have an official distribution channel:
        online selling, where buyers can directly buy the product online.

        %%%%%%%%%%%%%%%%%%%%%%%%%%%%%%%%%%%%

        \subsubsection*{Customer Relationships}

        The relationship with the customer does not end once they buy
        \textit{ec\textbf{\o}}, the company will create a phone line and email
        address so that the customers who are having issues with the product or
        need information requests can contact the firm directly. Therefore,
        personal assistance will be available along the whole purchase process,
        either during pre-sales and/or after sales situations.

        It will also create an online community, which allows for direct
        interactions among the company and its clients, where knowledge can be
        shared, and problems can be solved between different clients with the
        company as an intermediary. Through this platform, consumers could
        become what is known as co-creators of the product by suggesting ideas
        of upgrades that would make \textit{ec\textbf{\o}} more efficient.

        %%%%%%%%%%%%%%%%%%%%%%%%%%%%%%%%%%%%

    \subsection*{Infrastructure}

        \subsubsection*{Key Activities}

        The key activity required to execute \textit{Rainforest}'s value
        proposition are the technical development and programming of the project
        itself, both in terms of creating a product capable of detecting the
        plants' state and needs, and creating the website and the Alexa Skill.
        Another key activity would be to target \textit{ec\textbf{\o}}'s
        potential customers correctly by investing appropriately in the
        different channels needed to create awareness in consumers' minds that,
        hopefully, will end up in the product's purchase.

        %%%%%%%%%%%%%%%%%%%%%%%%%%%%%%%%

        \subsubsection*{Key Resources}

        In order to create \textit{ec\textbf{\o}} and sustan
        \textit{Rainforest}'s Business, physical components are needed such as
        the DHT22 to detect humidity and temperature, microcontrollers and
        boards such as the ESP-32 and water pumps such as the R385. Apart from
        that, human capital is crucial for the creation and correct functioning
        of the software that detect's the plant's current state, the creation of
        \textit{Alexa}'s skill and \textit{ec\textbf{\o}}'s website.

        %%%%%%%%%%%%%%%%%%%%%%%%%%%%%%%%

        \subsubsection*{Partner Network}

        Partnerships and alliances are crucial for a business, especially when
        the company is new to the marketplace. They optimize operations and
        reduce risks so that the company can focus on its core activity, which
        in \textit{Rainforest}'s case is the creation of \textit{ec\textbf{\o}}.

        The main and most important partnership for \textit{Rainforest} is the
        one with the physical distributors of their product. Even if the product
        is sold online, it still has to arrive to the client in order for them
        to benefit from it. Choosing the right partners is essential for the
        company to ensure that quality of product is maintained and the
        relationship with the client is kept honest and trustworthy.
    
        %%%%%%%%%%%%%%%%%%%%%%%%%%%%%%%%

    \subsection*{Finances}

        \subsubsection*{Cost structure}

        \textit{ec\textbf{\o}}'s business structure would be value-driven, since
        the enterprise will focus its efforts on creating value for its
        customers rather than minimizing costs as much as possible. It is clear
        that the company will try to minimize its costs to get higher revenues,
        but this is not the main objective for the firm at the time this
        document was being writter.

        In terms of costs, the key activities that will be more expensibe
        throughout the process are the technical development and computing
        programming of the project itself, both in terms of the low-level
        software and the multiple user interfaces. Whenever possible, especially
        when selling to businesses and governments, economies of scale could be
        reached, thus, decreasing the cost of each unit as the number of units
        that are ordered of the product increase.

        %%%%%%%%%%%%%%%%%%%%%%%%%%%%%%%%

        \subsubsection*{Revenue Streams}

        \textit{ec\textbf{\o}}'s customers would be willing to pay a premium
        compared to other competitors for the quality and value of the product.
        \textit{ec\textbf{\o}} offers possible new upgrades that are not out in
        the market yet, such as detecting if the plant is sick or what the
        disease that is causing it is. Other attributes for which customers
        would be willing to pay more are its complimentary website and the
        opportunity of connecting it to \textit{Amazon Alexa}.

        Taking a look at the market, the cost of material to create the product
        would be around \$30US.\ To this cost, other fixed costs such as the
        creation of the website, the design and implementation of the physical
        system itself and the \textit{Alexa} skill, must be added. Comparing it
        to the market, the most similar competitor would be ``Xiaomi Mi Smart
        Flowerpot Plant Pot'' which costs \$50US.\

        This fact provides \textit{Rainforest} with an idea of what price would
        be reasonable to ask for the product and still generate revenue, but
        taking into account that \textit{Xiaomi}'s product doesn't offer as many
        attributes as \textit{ec\textbf{\o}}, it would be reasonable to ask for
        approximately \$60US as the final price for the product. The revenue
        streams of the firm will mainly come from the sale of this very product
        and this price would give \textit{Rainforest} a decent revenue margin to
        ensure its success in the future.

    %%%%%%%%%%%%%%%%%%%%%%%%%%%%%%%%%%%%

    \subsection*{Marketing}

    When it comes to actually getting to the clients. There are many factors to
    take into account. From the way the company is perceived in the market, to
    the way customers see the product and the sentiment they have towards it. In
    order to maximize the labors of the company and market the product in the
    best way possible, \textit{Rainfores}'s marketing team has devised the
    following factors and elements to follow.

        %%%%%%%%%%%%%%%%%%%%%%%%%%%%%%%%

        \subsubsection*{Perception and differentiation}

        When it comes to \textit{Rainforest} itself, the focus and the angle the
        company hsould have towards its customers and towards the market could
        be characterized as that of the personality of a young technological
        activist that fights for the future of planet Earth. Some of the most
        important factors this company should have and the way it should behave
        are the following:

        \begin{easylist}[itemize]

        & \textbf{Young}. With \textit{young} the team does not mean the company
        should act childlishly, but with the desire to do something and the wish
        to work hard and earn people's trust.

        & \textbf{Connected}. Maintaining a connection with the customer, be it
        through social media or through a different online pressence is a key
        factor when appealing to the younger audiences and customers. The very
        inclusion of `technology' in the product will attract younger customers
        towards it and statistics show a company that keeps in touch is a
        company customers trust.

        & \textbf{Revolutionary}. This is not a term that defines the product or
        the way to do things itself. With a \textit{Rainforest} that is
        revolutionary what the marketing team means is a company that is not
        afraid to do things differently and a company that does not simply fear
        change for the sake of change.

        & \textbf{Responsible}. At the same time as young, connected and
        revolutionary, the company should always maintain it's status as a
        responsible one. The product \textit{Rainforest} sells is one that helps
        nature and keeps the planet clean. It's a product that helps people
        reduce wasted resources and keep plants alive, therefore one could say
        the product is responsible with the environment, and so should the
        company be.

        & \textbf{Involved}. An involved and present company is one that is sure
        to succeed, especially when it is related to nature and the topics
        \textit{Rainforest} treats and gets involved with. A company that is
        about the betterment of nature and the improvement of it \textit{must}
        be one that is involved.

        & \textbf{Honest}. This point is obvious. A company that does not lie
        and stays honest with their customers is a company that gains their
        trust. A company that inspires security and makes people feel confident
        when buying their products.

        \end{easylist}

%%%%%%%%%%%%%%%%%%%%%%%%%%%%%%%%%%%%%%%%%%%%%%%%%%%%%%%%%%%%%%%%%%%%%%%%%%%%%%%

\section*{Development Plan}

Given the very nature of the software development plan the team has cosen for
the project, the need for an actual `development plan' \textit{per se} is not of
the utmost importance. The team has instead chosen to divide the work in a
realistic set of project objectives and deliverable items within the provided
timeframe of the project.

It is true that projects generally function better with a somewhat defined
development workflow. This case, however, is not applicable to the engineering
team at \textit{Rainforest}. The very nature of their relationship and the way
members are able to collaborate seemlessly given their already based background
makes the \textit{Lean} software development methodology a great one for the
team. In fact, the addition and creation of a software development plan into the
development environment would take more time than the actual development itself.

Every single characteristic found in this document and defined as an objective
can be considered a deliverable item unles specified otherwise. This section is
also accompanied by a document called ``planning'' where a prelimiary idea and
conceptualization of how the work should be distributed among the work months
can be shown. This, however is a mere conceptualization of a preliminary idea,
it should not be taken as definitive.

%%%%%%%%%%%%%%%%%%%%%%%%%%%%%%%%%%%%%%%%%%%%%%%%%%%%%%%%%%%%%%%%%%%%%%%%%%%%%%%%

\section*{Risk Assessment}

%%%%%%%%%%%%%%%%%%%%%%%%%%%%%%%%%%%%%%%%%%%%%%%%%%%%%%%%%%%%%%%%%%%%%%%%%%%%%%%%

\section*{Contingency Plan}

%%%%%%%%%%%%%%%%%%%%%%%%%%%%%%%%%%%%%%%%%%%%%%%%%%%%%%%%%%%%%%%%%%%%%%%%%%%%%%%%

\section*{Overview of the Project}

%%%%%%%%%%%%%%%%%%%%%%%%%%%%%%%%%%%%%%%%%%%%%%%%%%%%%%%%%%%%%%%%%%%%%%%%%%%%%%%%

\printbibliography

\end{document}
